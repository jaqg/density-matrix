% \tableofcontents
% \listofnotes

Hasta ahora había estado haciendo el programa sobre la 1-RDM $D^{(1)}_{ij}$,
2-RDM $D^{(2)}_{ijkl}$, integrales monoelectrónicas sobre el hamiltoniano
$h_{ij}$ e integrales bielectrónicas $ \Braket{ij | kl}$ para un ejemplo
(una de las moléculas de \ch{H2O} del scan) para ver simplemente que funcionaba.

Ya lo tengo todo programado y, aunque funciona (obviamente todo compila bien),
no obtengo resultados siquiera similares a los que obtengo con DALTON o con
Gaussian09.

Aquí desarrollo a detalle todo lo que hago hasta calcular el primer valor
de energía.

Nota: añado en rojo dudas y/o notas en sitios donde creo que puede estar el error o 
fragmentos en los que tengo dudas.

Mi intención no es que leas todo el documento al detalle, ni mucho menos el 
código (que incluyo más bien para tenerlo a mano por si tengo que consultarte
algo), por eso indico con rojo las dudas.
Mi intención más bien es que me puedas corregir si tengo algún malentendido
conceptual o con Dalton.
