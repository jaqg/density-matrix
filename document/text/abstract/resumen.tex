% ------
\section*{Resumen}
% ------
En este Trabajo Fin de Máster, la aplicación de la Teoría del Funcional de la
Matriz de Densidad Reducida (RDMFT, por sus siglas en inglés) en el contexto de
los cálculos de estructura electrónica es explorada.
Se estudia la viabilidad y precisión resultantes del uso de las
matrices densidad reducida de primer orden (1-RDM) y segundo orden (2-RDM)
en lugar de la función de onda en los métodos tradicionales.
Tanto las RDMs exactas, calculadas a partir de su definición formal, como 
distintas RDMs aproximadas son estudiadas.
El objetivo es crear un entorno de trabajo que permita la optimización y 
verificación de una potencialmente nueva parametrización de la 2-RDM propuesta
por nuestro grupo y desarrollada en este documento.

% Una ventaja significativa de la RDMFT frente a los métodos convencionales
% basados en la función de onda es la eficiencia con la que los efectos de
% correlación estáticos son considerados, haciendo a la RDMFT una teoría prometedora
% en el tratamiento de sistemas con efectos de correlación estática notables.
Una ventaja significativa de la RDMFT frente a los métodos convencionales
basados en la función de onda es su capacidad para manejar eficazmente los 
efectos de correlación estática, lo que la convierte en un enfoque prometedor 
para sistemas donde estos efectos son pronunciados.
Además, el  uso de RDMs simplifica el tratamiento
del término de energía cinética y evita la necesidad de introducir un
sistema ficticio no-interactuante, como se requiere en la Teoría del Funcional de la
Densidad (DFT).

A pesar de su potencial, la aplicación de la RDMFT está actualmente limitada a 
sistemas pequeños debido a su complejidad computacional.
Esta limitación se deriva de la naturaleza compleja
de las RDMs de orden superior y de garantizar su
$N$-representabilidad.

Futuras líneas de investigación incluyen extender estas metodologías 
a la Teoría del Funcional de la Matriz de Densidad 
Reducida Dependiente del Tiempo (TDRDMFT), lo que 
permitiría el estudio de procesos dinámicos en sistemas electrónicos.
Además, se han realizado avances teóricos significativos en las últimas décadas,
particularmente con el desarrollo de las ecuaciones de Schrödinger contraídas y 
RDMs de orden superior, que prometen mejorar aún más 
la aplicabilidad y robustez de la RDMFT.

En conclusión, este trabajo proporciona un marco comprensivo para emplear la 
RDMFT en cálculos de estructura electrónica, demostrando sus ventajas 
y desafíos que deben abordarse para su aplicación más amplia.

\vspace{0.3cm}
\textbf{Objetivos:}
\begin{enumerate}[label={}]
    \item \textit{Primero:} Crear un programa integrado con el software Dalton 
        para la obtención de RDMs exactas.
    \item \textit{Segundo:} Validar algunas de las aproximaciones disponibles
        para el funcional de repulsión electrónica en términos de la 2-RDM.
    \item \textit{Tercero:} Desarrollar la hipótesis de una posible nueva 
        aproximación y proponer posibles métodos de parametrización y optimización.
    \item \textit{Cuarto:} Crear un entorno de trabajo para la construcción, prueba y 
        estudio profundo de la 2-RDM, con el cálculo de propiedades derivadas 
        de la RDM, como la energía, comparables con la literatura y/o otros cálculos, 
        y métricas directas, como la distancia de Minkowski, para 
        la comparación directa de RDMs.
\end{enumerate}

