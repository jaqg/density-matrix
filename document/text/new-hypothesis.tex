% ------------------------------
\section{\textcolor{red}{A possible new approximation}} % (fold)
\label{sec:new-hypothesis}
% ------------------------------

Because of Hohenberg-Kohn theorems, it should be possible to express $E_0$ in
terms of $\mat{\oedm}$.
Then, it should be true that $\mat{\tedm}$ can be written as 
\begin{equation} \label{eq:myaprox-2rdm-1}
    \tedm_{pqrs} = 
    \sum_{ijkl} C_{pqrs,ijkl} \oedm_{ij} \oedm_{kl}
    ,
\end{equation}
for a given coefficient $C_{pqrs,ijkl}$.

In principle, recalling \cref{eq:tedm-spin-representation}, a similar expression
for $d_{pqrs}$ should also hold
\begin{equation} \label{eq:myaprox-d-1}
    d_{pqrs} = 
    \sum_{ijkl} C_{pqrs,ijkl}^{\prime} \oedm_{ij} \oedm_{kl}
    .
\end{equation}

Then, assuming that we choose a basis diagonalizing $\mat{\oedm}$ 
\begin{equation} \label{eq:myaprox-diag-1rdm}
    \oedm_{pq} = \nu_p \delta_{pq}
    ,
\end{equation}
such that $d_{pqrs}$ in \cref{eq:myaprox-d-1} is rewritten as
\begin{equation} \label{eq:myaprox-d-2}
    d_{pqrs} = 
    \sum_{ijkl} C_{pqrs,ijkl}^{\prime} \nu_i \nu_k \delta_{ij} \delta_{kl}
    =
    \sum_{ik} C_{pqrs,iikk}^{\prime} \nu_i \nu_k
    .
\end{equation}

The energy, expanding the $\frac{1}{2} \PBraket{pq | rs}$ factor of
\cref{eq:GS-energy-1}, reads
\begin{equation}
    E_0 =
    % \sum_{pq} h_{pq} \oedm_{pq} +
    % \frac{1}{2} \sum_{pqrs} \PBraket{pq | rs}
    % \left( d_{pqrs} - \oedm_{ps} \delta_{qr} \right)
    % =
    \sum_{pq} h_{pq} \oedm_{pq} +
    \frac{1}{2} \sum_{pqrs} \PBraket{pq | rs} d_{pqrs}
    - 
    \frac{1}{2} \sum_{pqrs} \PBraket{pq | rs} \oedm_{ps} \delta_{qr}
    ,
\end{equation}
and with \cref{eq:myaprox-diag-1rdm,eq:myaprox-d-2}
\begin{equation}
    E_0 =
    \sum_{pq} h_{pq} \nu_p \delta_{pq} +
    \frac{1}{2} \sum_{pqrs} \sum_{ijkl} \PBraket{pq | rs}
    C_{pqrs,ijkl}^{\prime} \nu_i \nu_k \delta_{ij} \delta_{kl}
    - 
    \frac{1}{2} \sum_{pqrs} \PBraket{pq | rs} \nu_p \delta_{ps} \delta_{qr}
    .
\end{equation}

Therefore, the energy is given by
\begin{equation} \label{eq:myaprox-E0-1}
    E_0 =
    % \sum_{pq} h_{pq} \nu_p \delta_{pq} +
    % \frac{1}{2} \sum_{pqrs} \sum_{ijkl} \PBraket{pq | rs}
    % C_{pqrs,ijkl}^{\prime} \nu_i \nu_k \delta_{ij} \delta_{kl}
    % - 
    % \frac{1}{2} \sum_{pqrs} \PBraket{pq | rs} \oedm_{ps} \delta_{qr}
    % =
    \sum_{p} h_{pp} \nu_p +
    \frac{1}{2} \sum_{pqrs} \lambda_{pqrs} \PBraket{pq | rs}
    - 
    \frac{1}{2} \sum_{pq} \PBraket{pq | qp} \nu_p
    ,
\end{equation}
where 
\begin{equation}
    \lambda_{pqrs} = \sum_{ij} 
    C_{pqrs,iijj}^{\prime} \nu_i \nu_j
    .
\end{equation}

To this point, no approximation has been made. 
In fact, $\lambda_{pqrs}$ is simply $d_{pqrs}$ in the chosen natural basis.

A possible parametrization of $\mat{\lambda}$, proposed by Alfredo, is given as follows.

Looking at \cref{eq:myaprox-d-1}, in
which $ \mat{\tedm}$ is given by products
of $ \mat{\oedm}$'s, and knowing that the electron-electron interaction energy is
directly given by the product of $ \mat{\tedm}$ and the two-electron integrals
(see \cref{eq:Eee_from_Gamma}),
this energy functional could be splitted into Coulomb $ \PBraket{pp | qq}$ and
Exchange $ \PBraket{pq | qp}$ contributions.
And, so for, $\mat{\lambda}$ 
\begin{equation}
    \lambda_{pqrs} \PBraket{pq | rs}
    \leftarrow
    \tilde{\lambda}_{pqrs} \PBraket{pp | qq}
    +
    \tilde{\lambda}_{pqrs} \PBraket{pq | qp}
    .
\end{equation}

Then, a possible parametrization for $\mat{\lambda}$ might be 
\begin{equation}
    \lambda_{pqrs} =
    \alpha_{pqrs}
    -
    \beta_{pqrs}
    ,
\end{equation}
where $\alpha_{pqrs}$ refer to type products of two $ \mat{\oedm}$ for the
Coulomb contribution with given $\mu$ factors
\begin{equation}
    \alpha_{pqrs} = \mu_{pq} \oedm_{pq} \mu_{rs} \oedm_{rs} ,
\end{equation} 
and $\beta_{pqrs}$ to the exchange contributions
\begin{equation}
    \beta_{pqrs} = \mu_{ps}^{\prime} \oedm_{ps} \mu_{rq}^{\prime} \oedm_{rq}.
\end{equation} 

Therefore, in the natural basis, $ \mat{\lambda}$ reads
\begin{equation}
    \lambda_{pqrs} =
    \mu_{pr} \nu_p \nu_r \delta_{pq} \delta_{rs}
    -
    \mu_{pq}^{\prime} \nu_p \nu_q \delta_{ps} \delta_{qr}
    .
\end{equation}

Then, in \cref{eq:myaprox-E0-1}
\begin{equation}
    E_0 =
    % \sum_{p} h_{pp} \nu_p +
    % \frac{1}{2} \sum_{pqrs} \lambda_{pqrs} \PBraket{pq | rs}
    % - 
    % \frac{1}{2} \sum_{pq} \PBraket{pq | qp} \nu_p
    % =
    \sum_{p} h_{pp} \nu_p +
    \frac{1}{2} \sum_{pqrs} 
    \mu_{pr} \nu_p \nu_r \delta_{pq} \delta_{rs}
    \PBraket{pq | rs}
    -
    \frac{1}{2} \sum_{pqrs} 
    \mu_{pq}^{\prime} \nu_p \nu_q \delta_{ps} \delta_{qr}
    \PBraket{pq | rs}
    -
    \frac{1}{2} \sum_{pq} \PBraket{pq | qp} \nu_p
    ,
\end{equation}
and
\begin{equation}
    E_0 =
    \sum_{p} h_{pp} \nu_p +
    \frac{1}{2} \sum_{pq} 
    \mu_{pq} \nu_p \nu_q 
    \PBraket{pp | qq}
    -
    \frac{1}{2} \sum_{pq} 
    \mu_{pq}^{\prime} \nu_p \nu_q
    \PBraket{pq | qp}
    -
    \frac{1}{2} \sum_{pq} \PBraket{pq | qp} \nu_p
    .
\end{equation}

Therefore, the ground state energy for this parametrization is given by
\begin{equation}
    \therefore
    E_0 =
    \sum_{p} h_{pp} \nu_p +
    \frac{1}{2} \sum_{pq} 
    \nu_p \nu_q
    \left[ 
        \mu_{pq} \PBraket{pp | qq}
        -
        \mu_{pq}^{\prime} \PBraket{pq | pq}
    \right]
    -
    \frac{1}{2} \sum_{pq} 
    \PBraket{pq | pq} \nu_p
    ,
\end{equation}
and the 2-RDM, recalling \cref{eq:tedm-spin-representation}, by 
\begin{equation}
    \therefore
    \tedm_{pqrs} =
    % \lambda_{pqrs} - \delta_{qr} D_{ps} =
    \mu_{pr} \nu_p \nu_r \delta_{pq} \delta_{rs}
    -
    \mu_{pq}^{\prime} \nu_p \nu_q \delta_{ps} \delta_{qr}
    -
    \nu_p \delta_{qr} \delta_{ps}
    .
\end{equation}

This new parametrization for $\mat{\tedm}$ has to be optimized in order for
$\mat{\tedm}$ to be $N$-representable which, in the molecular
orbital basis, correspond to
\begin{itemize}
    \item Occupation numbers, $\nu_p$, are nonnegative and not greater than 2 
    \begin{equation}
        0 \le \nu_p \le 2\ \forall p,
    \end{equation}

    \item The normalization condition must be fulfilled, such that 
    \begin{equation}
        \sum_{i} \nu_i = n,
    \end{equation}
    and 
    \begin{equation}
        \sum_{ij} \tedm_{ijij} = n\left( n-1 \right), 
    \end{equation}

    \item $\mat{\tedm}$ must be positive-semidefinite, i.e. the quadratic form
        $\mat{v^{\dagger} \tedm v}$ must be possitive for any vector $\mat{v}$
        \begin{equation}
            \mat{\tedm} \succeq 0 \Leftrightarrow \Braket{\Psi | \tedm | \Psi} \ge 0,
        \end{equation}
        the eigenvalues of $\mat{\tedm}$ must be possitive

\end{itemize}

There are more complex conditions, as the two- and three-index conditions\mycite{garrod1964reduction}.
Here, the main definitions for the two-index conditions are given as\mycite{verstichel2012variational}:
\begin{itemize}
    \item The $\mathcal{I}$ condition: the probability of finding a 
        two-particle pair is larger than zero.

    \item The $\mathcal{Q}$ condition: the probability of finding a two-hole
        pair has to be larger than zero 
        \begin{equation}
            \mathcal{Q} \succeq 0, \quad
            \mathcal{Q}_{\alpha \beta;\gamma \delta} =
            \sum_{i} \omega_i 
            \Braket{\Psi_i | a_{\alpha} a_{\beta} 
            a_{\delta}^{\dagger} a_{\gamma}^{\dagger} | \Psi_i},
        \end{equation}
        which can be rewritten as the followin linear matrix mapping 
        \begin{equation}
            \mathcal{Q}\left( \sordm \right)_{\alpha\beta;\gamma\delta} =
            \delta_{\alpha\gamma}\delta_{\beta\delta} - 
            \delta_{\beta\gamma}\delta_{\alpha\delta} + 
            \sordm_{\alpha\beta;\gamma\delta} - 
            \left(
                \delta_{\alpha\gamma}\fordm_{\beta\delta} - 
                \delta_{\alpha\delta}\fordm_{\beta\gamma} - 
                \delta_{\beta\gamma}\fordm_{\alpha\delta} + 
                \delta_{\beta\delta}\fordm_{\alpha\gamma}
            \right)
            ,
        \end{equation}
        which, together with the $\mathcal{I}$ condition, already insures the 
        Pauli principle.

    \item The $\mathcal{G}_1$ condition: the probability of finding a 
        particle-hole pair must be larger than zero
        \begin{equation}
            \mathcal{G}_1 \succeq 0, \quad
            \left( \mathcal{Q}_1 \right)_{\alpha \beta;\gamma \delta} =
            \sum_{i} \omega_i 
            \Braket{\Psi_i | a_{\alpha}^{\dagger} a_{\beta} 
            a_{\delta}^{\dagger} a_{\gamma} | \Psi_i},
        \end{equation}
        which can be rewritten using anticommutation relations as
        \begin{equation}
            \mathcal{G}_1\left( \sordm \right)_{\alpha\beta;\gamma\delta} =
            \delta_{\beta \delta} \fordm_{\alpha \gamma} - \sordm_{\alpha \delta ; \gamma \beta}.
            .
        \end{equation}

        The combined conditions $\mathcal{IQG_1}$ are known as the standard
        two-index conditions, which already lead to very good approximations
        form some systems.

    \item The $\mathcal{G}_2$ condition: the probability of finding a 
        hole-particle pair must be positive
        \begin{equation}
            \mathcal{G}_2 \succeq 0, \quad
            \left( \mathcal{Q}_2 \right)_{\alpha \beta;\gamma \delta} =
            \sum_{i} \omega_i 
            \Braket{\Psi_i | a_{\alpha} a_{\beta}^{\dagger} 
            a_{\delta} a_{\gamma}^{\dagger} | \Psi_i},
        \end{equation}
        as a function of $\sordm$
        \begin{equation}
            \mathcal{G}_2\left( \sordm \right)_{\alpha\beta;\gamma\delta} =
            \delta_{\alpha \beta } \delta_{\gamma \delta} - 
            \delta_{\alpha \beta } \fordm_{\gamma \delta} - 
            \delta_{\gamma \delta } \fordm_{\alpha \beta} +
            \delta_{\alpha \gamma } \fordm_{\beta \delta} -
            \sordm_{\alpha \delta ; \gamma \beta}
            .
        \end{equation}
\end{itemize}

There are stronger conditions as the three-index conditions $\mathcal{T}_1 \mathcal{T}_2 \mathcal{T}_3$,
the primed conditions\mycite{garrod1964reduction,mazziotti2007reduced,mazziotti2006variational}
$\mathcal{G^{\prime}} \mathcal{T}_2^{\prime}$, and non-standard conditions\mycite{verstichel2012variational}.


This optimization would, idially, be done formally.
However, it can be done variationally such that the variational parameters
$\mu,\mu^{\prime}$ minimize the energy.
This method, refered as v2DM, can be formally defined as a semidesemidefinite
programming following the $N$-representability constraints\mycite{verstichel2012variational}.

Also, it can be done finding a function 
$Q\left( \nu_p, \nu_q, \nu_r, \nu_s \right)$ for the factors $\mu,\mu^{\prime}$
depending on occupation numbers using, for example,
a Padé approximant or any other approximant.
A general Padé approximant of order $\left[ m / n \right]$ around a point 
$x=\nu_p \nu_q \nu_r \nu_s$ is the rational function 
\begin{equation}
    Q\left( \nu_p, \nu_q, \nu_r, \nu_s \right) =
    \frac{
        \sum_{j = 0}^{m} a_j \nu_p^{j} \nu_q^{j} \nu_r^{j} \nu_s^{j}
    }{
        1 + \sum_{k=1}^{n} b_k \nu_p^{k} \nu_q^{k} \nu_r^{k} \nu_s^{k}
    } =
    \frac{
        a_0 + a_1 \nu_p \nu_q \nu_r \nu_s + \cdots + a_m \nu_p^{m} \nu_q^{m} \nu_r^{m} \nu_s^{m}
    }{
        1 + b_1 \nu_p \nu_q \nu_r \nu_s + \cdots + b_n \nu_p^{n} \nu_q^{n} \nu_r^{n} \nu_s^{n}
    }
    .
\end{equation}
This approach has already been used to find a fully empirically function by Marques
and Lathiotakis (ML)\mycite{marques2008empirical} using a Padé approximant
depending on a variable $x = n_p n_q$.

