% ------------------------------
\section{\textcolor{red}{New hypothesis}} % (fold)
\label{sec:new-hypothesis}
% ------------------------------

Because of Hohenberg-Kohn theorems, it should be possible to express $E_0$ in
terms of $\mat{\oedm}$.
Then, it should be true that $\mat{\tedm}$ can be written as 
\begin{equation} \label{eq:myaprox-2rdm-1}
    \tedm_{pqrs} = 
    \sum_{ijkl} C_{pqrs,ijkl} \oedm_{ij} \oedm_{kl}
    .
\end{equation}

In principle, recalling \cref{eq:tedm-spin-representation}, a similar expression
for $d_{pqrs}$ should also hold
\begin{equation} \label{eq:myaprox-d-1}
    d_{pqrs} = 
    \sum_{ijkl} C_{pqrs,ijkl}^{\prime} \oedm_{ij} \oedm_{kl}
    .
\end{equation}

Then, assuming that we choose a basis diagonalizing $\mat{\oedm}$ 
\begin{equation} \label{eq:myaprox-diag-1rdm}
    \oedm_{pq} = \nu_p \delta_{pq}
    ,
\end{equation}
such that $d_{pqrs}$ in \cref{eq:myaprox-d-1} is rewritten as
\begin{equation} \label{eq:myaprox-d-2}
    d_{pqrs} = 
    \sum_{ijkl} C_{pqrs,ijkl}^{\prime} \nu_i \nu_k \delta_{ij} \delta_{kl}
    =
    \sum_{ik} C_{pqrs,iikk}^{\prime} \nu_i \nu_k
    .
\end{equation}

The energy, expanding the $\frac{1}{2} \PBraket{pq | rs}$ factor of
\cref{eq:GS-energy-1}, reads
\begin{equation}
    E_0 =
    % \sum_{pq} h_{pq} \oedm_{pq} +
    % \frac{1}{2} \sum_{pqrs} \PBraket{pq | rs}
    % \left( d_{pqrs} - \oedm_{ps} \delta_{qr} \right)
    % =
    \sum_{pq} h_{pq} \oedm_{pq} +
    \frac{1}{2} \sum_{pqrs} \PBraket{pq | rs} d_{pqrs}
    - 
    \frac{1}{2} \sum_{pqrs} \PBraket{pq | rs} \oedm_{ps} \delta_{qr}
    \right)
    ,
\end{equation}
and with \cref{eq:myaprox-diag-1rdm,eq:myaprox-d-2}
\begin{equation}
    E_0 =
    \sum_{pq} h_{pq} \nu_p \delta_{pq} +
    \frac{1}{2} \sum_{pqrs} \sum_{ijkl} \PBraket{pq | rs}
    C_{pqrs,ijkl}^{\prime} \nu_i \nu_k \delta_{ij} \delta_{kl}
    - 
    \frac{1}{2} \sum_{pqrs} \PBraket{pq | rs} \nu_p \delta_{ps} \delta_{qr}
    .
\end{equation}

Therefore, the energy is given by
\begin{equation} \label{eq:myaprox-E0-1}
    E_0 =
    % \sum_{pq} h_{pq} \nu_p \delta_{pq} +
    % \frac{1}{2} \sum_{pqrs} \sum_{ijkl} \PBraket{pq | rs}
    % C_{pqrs,ijkl}^{\prime} \nu_i \nu_k \delta_{ij} \delta_{kl}
    % - 
    % \frac{1}{2} \sum_{pqrs} \PBraket{pq | rs} \oedm_{ps} \delta_{qr}
    % =
    \sum_{p} h_{pp} \nu_p +
    \frac{1}{2} \sum_{pqrs} \lambda_{pqrs} \PBraket{pq | rs}
    - 
    \frac{1}{2} \sum_{pq} \PBraket{pq | qp} \nu_p
    ,
\end{equation}
where 
\begin{equation}
    \lambda_{pqrs} = \sum_{ij} 
    C_{pqrs,iijj}^{\prime} \nu_i \nu_j
    .
\end{equation}

To this point, no approximation has been made. 
In fact, $\lambda_{pqrs}$ is simply $d_{pqrs}$ in the chosen natural basis.

A possible parametrization of $\lambda$ proposed by Alfredo might be 
\begin{equation}
    \lambda_{pqrs} =
    \alpha_{pq} \alpha_{rs}
    -
    \beta_{ps} \beta_{qr}
    ,
\end{equation}
\textcolor{red}{¿$\alpha_{pq} \equiv \fordm_{pq}^{\alpha}$?}

that, in the natural basis reads
\begin{equation}
    \lambda_{pqrs} =
    \mu_{pr} \nu_p \nu_r \delta_{pq} \delta_{rs}
    -
    \mu_{pq}^{\prime} \nu_p \nu_q \delta_{ps} \delta_{qr}
    ,
\end{equation}

Then, in \cref{eq:myaprox-E0-1}
\begin{equation}
    E_0 =
    % \sum_{p} h_{pp} \nu_p +
    % \frac{1}{2} \sum_{pqrs} \lambda_{pqrs} \PBraket{pq | rs}
    % - 
    % \frac{1}{2} \sum_{pq} \PBraket{pq | qp} \nu_p
    % =
    \sum_{p} h_{pp} \nu_p +
    \frac{1}{2} \sum_{pqrs} 
    \mu_{pr} \nu_p \nu_r \delta_{pq} \delta_{rs}
    \PBraket{pq | rs}
    -
    \frac{1}{2} \sum_{pqrs} 
    \mu_{pq}^{\prime} \nu_p \nu_q \delta_{ps} \delta_{qr}
    \PBraket{pq | rs}
    -
    \frac{1}{2} \sum_{pq} \PBraket{pq | qp} \nu_p
    ,
\end{equation}
and
\begin{equation}
    E_0 =
    \sum_{p} h_{pp} \nu_p +
    \frac{1}{2} \sum_{pq} 
    \mu_{pq} \nu_p \nu_q 
    \PBraket{pp | qq}
    -
    \frac{1}{2} \sum_{pq} 
    \mu_{pq}^{\prime} \nu_p \nu_q
    \PBraket{pq | qp}
    -
    \frac{1}{2} \sum_{pq} \PBraket{pq | qp} \nu_p
    .
\end{equation}

Therefore, the ground state energy for this parametrization is given by
\begin{equation}
    \therefore
    E_0 =
    \sum_{p} h_{pp} \nu_p +
    \frac{1}{2} \sum_{pq} 
    \nu_p \nu_q
    \left[ 
        \mu_{pq} \PBraket{pp | qq}
        -
        \mu_{pq}^{\prime} \PBraket{pq | pq}
    \right]
    -
    \frac{1}{2} \sum_{pq} 
    \PBraket{pq | pq} \nu_p
    ,
\end{equation}
and the 2-RDM by 
\begin{equation}
    \therefore
    \tedm_{pqrs} =
    \mu_{pr} \nu_p \nu_r \delta_{pq} \delta_{rs}
    -
    \mu_{pq}^{\prime} \nu_p \nu_q \delta_{ps} \delta_{qr}
    -
    \nu_p \delta_{qr} \delta_{ps}
    .
\end{equation}

% TODO: comentar N-representabilidad en base MO
