% ------------------------------
\section{A possible parametrization of the 2-RDM} % (fold)
\label{sec:new-hypothesis}
% ------------------------------

Because of Hohenberg-Kohn theorems, it should be possible to express $E_0$ in
terms of $\mat{\oedm}$.
Then, it should be true that $\mat{\tedm}$ can be written as 
\begin{equation} \label{eq:myaprox-2rdm-1}
    \tedm_{pqrs} = 
    \sum_{ijkl} C_{pqrs,ijkl} \oedm_{ij} \oedm_{kl}
    ,
\end{equation}
for a given coefficient $C_{pqrs,ijkl}$.

In principle, recalling \cref{eq:tedm-spin-representation}, a similar expression
for $d_{pqrs}$ should also hold
\begin{equation} \label{eq:myaprox-d-1}
    d_{pqrs} = 
    \sum_{ijkl} C_{pqrs,ijkl}^{\prime} \oedm_{ij} \oedm_{kl}
    .
\end{equation}

Then, assuming that we choose a basis diagonalizing $\mat{\oedm}$ 
\begin{equation} \label{eq:myaprox-diag-1rdm}
    \oedm_{pq} = \nu_p \delta_{pq}
    ,
\end{equation}
such that $d_{pqrs}$ in \cref{eq:myaprox-d-1} is rewritten as
\begin{equation} \label{eq:myaprox-d-2}
    d_{pqrs} = 
    \sum_{ijkl} C_{pqrs,ijkl}^{\prime} \nu_i \nu_k \delta_{ij} \delta_{kl}
    =
    \sum_{ik} C_{pqrs,iikk}^{\prime} \nu_i \nu_k
    .
\end{equation}

The energy, expanding the $\frac{1}{2} \PBraket{pq | rs}$ factor of
\cref{eq:GS-energy-1}, reads
\begin{equation}
    E_0 =
    % \sum_{pq} h_{pq} \oedm_{pq} +
    % \frac{1}{2} \sum_{pqrs} \PBraket{pq | rs}
    % \left( d_{pqrs} - \oedm_{ps} \delta_{qr} \right)
    % =
    \sum_{pq} h_{pq} \oedm_{pq} +
    \frac{1}{2} \sum_{pqrs} \PBraket{pq | rs} d_{pqrs}
    - 
    \frac{1}{2} \sum_{pqrs} \PBraket{pq | rs} \oedm_{ps} \delta_{qr}
    ,
\end{equation}
and with \cref{eq:myaprox-diag-1rdm,eq:myaprox-d-2}
\begin{equation}
    E_0 =
    \sum_{pq} h_{pq} \nu_p \delta_{pq} +
    \frac{1}{2} \sum_{pqrs} \sum_{ijkl} \PBraket{pq | rs}
    C_{pqrs,ijkl}^{\prime} \nu_i \nu_k \delta_{ij} \delta_{kl}
    - 
    \frac{1}{2} \sum_{pqrs} \PBraket{pq | rs} \nu_p \delta_{ps} \delta_{qr}
    .
\end{equation}

Therefore, the energy is given by
\begin{equation} \label{eq:myaprox-E0-1}
    E_0 =
    % \sum_{pq} h_{pq} \nu_p \delta_{pq} +
    % \frac{1}{2} \sum_{pqrs} \sum_{ijkl} \PBraket{pq | rs}
    % C_{pqrs,ijkl}^{\prime} \nu_i \nu_k \delta_{ij} \delta_{kl}
    % - 
    % \frac{1}{2} \sum_{pqrs} \PBraket{pq | rs} \oedm_{ps} \delta_{qr}
    % =
    \sum_{p} h_{pp} \nu_p +
    \frac{1}{2} \sum_{pqrs} \lambda_{pqrs} \PBraket{pq | rs}
    - 
    \frac{1}{2} \sum_{pq} \PBraket{pq | qp} \nu_p
    ,
\end{equation}
where 
\begin{equation}
    \lambda_{pqrs} = \sum_{ij} 
    C_{pqrs,iijj}^{\prime} \nu_i \nu_j
    .
\end{equation}

To this point, no approximation has been made. 
In fact, $\lambda_{pqrs}$ is simply $d_{pqrs}$ in the chosen natural basis.

A possible parametrization of $\mat{\lambda}$, proposed by our group, is given as follows.

Looking at \cref{eq:myaprox-d-1}, in
which $ \mat{\tedm}$ is given by products
of $ \mat{\oedm}$'s, and knowing that the electron-electron interaction energy is
directly given by the product of $ \mat{\tedm}$ and the two-electron integrals
(see \cref{eq:Eee_from_Gamma}),
this energy functional could be splitted into Coulomb $ \PBraket{pp | qq}$ and
Exchange $ \PBraket{pq | qp}$ contributions.
And, so for, $\mat{\lambda}$ 
\begin{equation}
    \lambda_{pqrs} \PBraket{pq | rs}
    \leftarrow
    \tilde{\lambda}_{pqrs} \PBraket{pp | qq}
    +
    \tilde{\lambda}_{pqrs} \PBraket{pq | qp}
    .
\end{equation}

Then, a possible parametrization for $\mat{\lambda}$ might be 
\begin{equation}
    \lambda_{pqrs} =
    \alpha_{pqrs}
    -
    \beta_{pqrs}
    ,
\end{equation}
where $\alpha_{pqrs}$ refer to type products of two $ \mat{\oedm}$ for the
Coulomb contribution with given $\mu$ factors
\begin{equation}
    \alpha_{pqrs} = \mu_{pq} \oedm_{pq} \mu_{rs} \oedm_{rs} ,
\end{equation} 
and $\beta_{pqrs}$ to the exchange contributions
\begin{equation}
    \beta_{pqrs} = \mu_{ps}^{\prime} \oedm_{ps} \mu_{rq}^{\prime} \oedm_{rq}.
\end{equation} 

Therefore, in the natural basis, $ \mat{\lambda}$ reads
\begin{equation}
    \lambda_{pqrs} =
    \mu_{pr} \nu_p \nu_r \delta_{pq} \delta_{rs}
    -
    \mu_{pq}^{\prime} \nu_p \nu_q \delta_{ps} \delta_{qr}
    .
\end{equation}

Then, in \cref{eq:myaprox-E0-1}
\begin{equation}
    E_0 =
    % \sum_{p} h_{pp} \nu_p +
    % \frac{1}{2} \sum_{pqrs} \lambda_{pqrs} \PBraket{pq | rs}
    % - 
    % \frac{1}{2} \sum_{pq} \PBraket{pq | qp} \nu_p
    % =
    \sum_{p} h_{pp} \nu_p +
    \frac{1}{2} \sum_{pqrs} 
    \mu_{pr} \nu_p \nu_r \delta_{pq} \delta_{rs}
    \PBraket{pq | rs}
    -
    \frac{1}{2} \sum_{pqrs} 
    \mu_{pq}^{\prime} \nu_p \nu_q \delta_{ps} \delta_{qr}
    \PBraket{pq | rs}
    -
    \frac{1}{2} \sum_{pq} \PBraket{pq | qp} \nu_p
    ,
\end{equation}
and
\begin{equation}
    E_0 =
    \sum_{p} h_{pp} \nu_p +
    \frac{1}{2} \sum_{pq} 
    \mu_{pq} \nu_p \nu_q 
    \PBraket{pp | qq}
    -
    \frac{1}{2} \sum_{pq} 
    \mu_{pq}^{\prime} \nu_p \nu_q
    \PBraket{pq | qp}
    -
    \frac{1}{2} \sum_{pq} \PBraket{pq | qp} \nu_p
    .
\end{equation}

Therefore, the ground state energy for this parametrization is given by
\begin{equation}
    \therefore
    E_0 =
    \sum_{p} h_{pp} \nu_p +
    \frac{1}{2} \sum_{pq} 
    \nu_p \nu_q
    \left[ 
        \mu_{pq} \PBraket{pp | qq}
        -
        \mu_{pq}^{\prime} \PBraket{pq | pq}
    \right]
    -
    \frac{1}{2} \sum_{pq} 
    \PBraket{pq | pq} \nu_p
    ,
\end{equation}
and the 2-RDM, recalling \cref{eq:tedm-spin-representation}, by 
\begin{equation} \label{eq:AQ-approximation}
    \therefore
    \tedm_{pqrs} =
    % \lambda_{pqrs} - \delta_{qr} D_{ps} =
    \mu_{pr} \nu_p \nu_r \delta_{pq} \delta_{rs}
    -
    \mu_{pq}^{\prime} \nu_p \nu_q \delta_{ps} \delta_{qr}
    -
    \nu_p \delta_{qr} \delta_{ps}
    .
\end{equation}

This new parametrization for $\mat{\tedm}$ has to be optimized in order for
$\mat{\tedm}$ to be $N$-representable which, in the molecular
orbital (MO) basis, correspond -at least- to
\begin{itemize}
    \item Occupation numbers, $\nu_p$, are nonnegative and not greater than 2 
        \begin{equation}
            0 \le \nu_p \le 2\ \forall p,
        \end{equation}

    \item The normalization condition must be fulfilled, such that 
        \begin{equation} \label{eq:MO-norm-cond-1}
            \sum_{i} \nu_i = n,
        \end{equation}
        and 
        \begin{equation} \label{ec:normalization-condition-2RDM}
            \sum_{ij} \tedm_{ijij} = \frac{n\left( n-1 \right)}{2}, 
        \end{equation}

    \item $\mat{\tedm}$ must be positive-semidefinite, i.e. the quadratic form
        $\mat{v^{\dagger} \tedm v}$ must be positive for any vector $\mat{v}$
        \begin{equation}
            \mat{\tedm} \succeq 0 \Leftrightarrow \Braket{\Psi | \mat{\tedm} | \Psi} \ge 0,
        \end{equation}
        and the eigenvalues of $\mat{\tedm}$ must be positive

\end{itemize}
More conditions as $\mathcal{I} \mathcal{Q} \mathcal{G}_1$ could be
considered for a better optimization.

This optimization would, ideally, be done formally.
The objective is to define $\mu, \mu^{\prime}$ such that the energy is minimized
while the $N$-representability conditions are satisfied.
One approach is to decompose the 2-RDM $\sordm_{pqrs}$ using any higher-order
singular value decomposition (HOSVD) method as, for example,
the Tucker decomposition\mycite{hitchcock1927expression, tucker1966mathematical}.
Also, the packed matrix $\tilde{\sordm}_{pq,rs}$ with composite indexes $pq, rs$
can be decomposed with any singular value decomposition (SVD) method as the Cholesky
decomposition\mycite{1924note}.

The optimization can be done variationally taking $\mu,\mu^{\prime}$ as variational
parameters, which can be formally defined as a semidefinite
programming (SDP) problem following the $N$-representability 
constraints.
Although another formulation has been used in this document, the optimization
problem is, basically, the v2DM problem\mycite{verstichel2012variational}.

Therefore, the algorithm \cref{al:2rdm-opt} is proposed.
For a given set of orbitals, a 2-RDM is obtained with \cref{eq:AQ-approximation}
for the $\mu, \nu$ parameters, along with the MO coefficients, $C$.
Then
\begin{enumerate}
    \item $\mu, \nu$ are optimized such that the 2-RDM is constrained
        to the nomalization conditions \cref{eq:MO-norm-cond-1,ec:normalization-condition-2RDM}.

    \item Purification.

        Since, to this point, the $N$-representability conditions are not
        fulfilled, the 2-RDM is decomposed with the Cholesky decomposition
        algorithm until a block with negative diagonal elements is found,
        which correspond to the negative semidefinite block of 2-RDM.
        Therefore, the rest correspond to a new positive semidefinite 2-RDM,
        which must be re-scaled and normalized.

        The 2-RDM is not positive semidefinite in the MO basis.
        Therefore, the purification process could very probably be more
        effective in the spin-orbital basis.

    \item Reoptimize the orbitals with the rest of $\mu, \nu$ fixed.
\end{enumerate}

\begin{algorithm}
    \caption{2-RDM optimization algorithm.}\label{al:2rdm-opt}
    \begin{algorithmic}[1]
        \Require $n$, $\mu$, $\nu$, $ \mat{C}_{\text{MO}}$
        % \Ensure $output$

        \For {$\mu_p \in \left\{ \mu \right\}$, $\nu_p \in \left\{ \nu \right\}$}
        \While {$\hat{\sordm} \not= n\left( n-1 \right) /2$ }
        \State Build $ \mat{\sordm}$ with $\mu_p$, $\nu_p$
        \State Compute the norm $ \hat{\sordm} = \sum_{ij} \sordm_{ijij} $
        \State Vary $\mu_p$, $\nu_p$ constrained to $ \nu_p + \sum_{i \not= p} \nu_i = n$
        \EndWhile
        \State Decompose 
        $\mat{\sordm} = \mat{L} \mat{L}^{T} \implies \sordm_{ij,kl} = \sum_{J} L_{ij}^{J} L_{kl}^{J}$
        by Cholesky:
        \For {$J=1$, \inline{range(}$\mat{\sordm}$\inline{)}}
        \State Compute $L_{ij}^{J} = \sordm_{ij,J}^{\left( J-1 \right)} \left( \sordm_{J,J}^{\left( J-1 \right)} \right)^{-1 /2}$
        \State Update $ \mat{\sordm}^{\left( J \right)} = \mat{\sordm}^{\left( J-1 \right)} - \mat{L}^{J} \left( \mat{L}^{J} \right)^{T}$
        \If {$\sordm_{J+1,J+1}^{\left( J \right)}$ < 0}
        \State Reorder $ \mat{\sordm}^{\left( J \right)}$ to move
        $\sordm_{J+1,J+1}^{\left( J \right)}$ into the next diagonal
        element
        \EndIf
        \EndFor
        \State Extract the positive semidifinte block of $ \mat{\sordm}$, $ \mat{\tilde{\sordm}}$
        \State Scale $ \mat{\tilde{\sordm}}$ to full dimension and re-normalize, $ \mat{\sordm} \leftarrow \mat{\tilde{\sordm}}$
        \State Re-optimize $ \mat{C}_{\text{MO}}$ with fixed 
        $\mu_i \in \left\{ \mu \right\}\ \forall_{i \not= p}$
        and
        $\nu_i \in \left\{ \nu \right\}\ \forall_{i \not= p}$
        \EndFor

        \State \textbf{return} $ \mat{\sordm}$
    \end{algorithmic}
\end{algorithm}

Basically, it results in a HF-like optimization.


Also, it can be solved finding a function 
$Q\left( \nu_p, \nu_q, \nu_r, \nu_s \right)$ for the factors $\mu,\mu^{\prime}$
depending on occupation numbers using, for example,
a Padé approximant or any other approximant.

A general Padé approximant of order $\left[ m / n \right]$ around a point 
$x=\nu_p \nu_q \nu_r \nu_s$ is the rational function 
\begin{equation}
    Q\left( \nu_p, \nu_q, \nu_r, \nu_s \right) =
    \frac{
        \sum_{j = 0}^{m} a_j \nu_p^{j} \nu_q^{j} \nu_r^{j} \nu_s^{j}
    }{
        1 + \sum_{k=1}^{n} b_k \nu_p^{k} \nu_q^{k} \nu_r^{k} \nu_s^{k}
    } =
    \frac{
        a_0 + a_1 \nu_p \nu_q \nu_r \nu_s + \cdots + a_m \nu_p^{m} \nu_q^{m} \nu_r^{m} \nu_s^{m}
    }{
        1 + b_1 \nu_p \nu_q \nu_r \nu_s + \cdots + b_n \nu_p^{n} \nu_q^{n} \nu_r^{n} \nu_s^{n}
    }
    ,
\end{equation}
which is the best approximation to $Q$ near $x$ of a given order, as
the approximant's power series agrees with the power series of the function
it is approximating.

This approach has already been used to find a fully empirical function by Marques
and Lathiotakis (ML)\mycite{marques2008empirical} using a $[1/ 1]$ Padé approximant
depending on a variable $x = n_p n_q$
\begin{equation}
    f^{\text{ML}}\left( n_j, n_k \right) =
    x
    \frac{
        a_0 + a_1 x
    }{
        1 + b_1 x
    }
    ,
\end{equation}
where the parameters $a_1$ and $b_1$ are optimized to minimize the energy.
The function is multiplied by $x$ to ensure that the
contribution of completely empty states $\left( n_j = 0 \right)$ to the
exchange-correlation energy is zero.
The functional is also forced to $f^{\text{ML}}\left( 1 \right) = 1$ by
setting $a_0 = 1 + b_1 - a_1$ to recover the Hartree-Fock limit in the
case of an idempotent density matrix.

A similar algorithm to \cref{al:2rdm-opt} could be use to optimize
$a_1$ and $b_1$.
