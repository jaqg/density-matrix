\graphicspath{{./figures/}}

% ------
\section{Introduction}
% ------
% NOTAS SACADAS DEL PERNAL
Functionals of one-electron reduced density matrix (1-RDDM), $\gamma$, defined
for an $N$-electron wavefunction, $\Psi$, as

\begin{equation} \label{ec:self-adjointness-gamma}
    \gamma \left( \mat{x}, \mat{x}\tdprime \right) =
    N \idotsint
    \Psi \left( \mat{x}, \mat{x_2}, \ldots,  \mat{x}_N \right)
    \Psi^{*} \left( \mat{x}\tdprime, \mat{x_2}, \ldots,  \mat{x}_N \right)
    \dd{ \mat{x}_2} \cdots \dd{ \mat{x}_N}
    ,
\end{equation}
where $ \mat{x} = \left( \mat{r}, s \right)$ is a combined spatial and spin
coordinate.

As inmidiate advantage of using 1-RDDM instead of the electron density, $\rho$,
is that the kinetic energy is an explicit functional of $\gamma$ but not of
$\rho$.
Then, there is no need to introduce a ficticious noninteracting system.
Moreover, orbitals present in RDMFT are fractionally occupied, so functionals 
of $\gamma$ seem to be better suited to account for static correlation.

Self-adjointness of $\gamma$, as defined in \cref{ec:self-adjointness-gamma},
allows for its spectral representation\mycite{lowdin1955quantum}
\begin{equation}
    \gamma \left( \mat{x}, \mat{x}\tdprime \right) =
    \sum_{p} n_p 
    \varphi_p \left( \mat{x} \right)
    \varphi_p^{*} \left( \mat{x}\tdprime \right)
    .
\end{equation}

The eigenvalues of 1-RDM are called natural occupation numbers,
$\left\{ n_p \right\}$, and the eigenfunctions are known as natural
spinorbitals, $\left\{ \varphi_p \right\}$.
For convention, the indices $p,q,r,s$ are referred to natural spinorbitals
and $a,b,c,d$ to arbitrary one-eletron functions.
