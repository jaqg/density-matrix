\graphicspath{{./figures/}}

% ------
\section{Introduction}
% ------
In the document, the second quantization formalism is used. 
Consequently, some useful definitions\mycite{helgaker2000second} are recalled
in this section.
Additionally, atomic units are used all throughout the document.

% ------
\subsection{Second quantization} % (fold)
\label{sec:Second quantization}

The occupation-number (ON) operators are defined for the sets
of creation and annihilation operators, $\left\{ a^{\dagger} \right\}$, 
$\left\{ a \right\}$, as
\begin{equation}
    n_p = a_p^{\dagger} a_p
    .
\end{equation}

For a given ON vector $\Ket{k}$, the occupation number $k_p$ is obtained by
counting the number of electrons in spin orbital $p$
\begin{equation}
    n_p \Ket{k} =
    a_p^{\dagger} a_p \Ket{k} =
    k_p \Ket{k}
    .
\end{equation}

The ON operators are Hermitian 
\begin{equation}
    n_p^{\dagger} =
    \left( a_p^{\dagger} a_p \right)^{\dagger} =
    a_p^{\dagger} a_p =
    n_p
    ,
\end{equation}
commute among themselves 
\begin{equation}
    n_p n_q \Ket{k} =
    k_p k_q \Ket{k} =
    k_q k_p \Ket{k} =
    n_q n_p \Ket{k}
    ,
\end{equation}
and,
since in the spin-orbital basis the ON operators are projection operators,
they are idempotent, 
\begin{equation}
    n_p^2 =
    n_p n_p =
    a_p^{\dagger} a_p a_p^{\dagger} a_p =
    a_p^{\dagger} \left( 1 - a_p^{\dagger} a_p \right) a_p =
    a_p^{\dagger} a_p =
    n_p
    .
\end{equation}
Therefore, its eigenvalues can only be 0 or 1.

The particle-number operator, or simply the number operator, is the Hermitian
operator resulting from adding together all ON operators in the Fock space 
\begin{equation}
    \hat{N} =
    \sum_{p=1}^{m} a_p^{\dagger} a_p
    ,
\end{equation}
which returns the number of electrons in an ON vector 
\begin{equation}
    \hat{N} \Ket{k} =
    \sum_{p=1}^{m} k_p \Ket{k} =
    n \Ket{k}
    .
\end{equation}

Also, the $\delta_{pq}$ operator is defined as
\begin{equation}
    \delta_{pq} = a_p a_q^{\dagger} + a_q^{\dagger} a_p
    ,
\end{equation} 
which is null if $p \not= q$.

% subsubsection Second quantization (end)

\subsection{Reduced Density Matrix Functional Theory} % (fold)

% subsubsection 1-RDM (end)
Many methods of electronic structure theory are based on variational
optimization of an energy functional.
An idempotent first-order reduced density matrix (1-RDM) is variationally
optimized in the Hartree-Fock (HF) method\mycite{kollmar2004structure}.

The 1-RDM is determined by the natural orbitals and their occupation
numbers\mycite{lowdin1955quantuma}, leading to denote the corresponding energy
functional as density matrix functional (DMF) or natural orbital functional (NOF).
This is explained in detail in the following sections.

% \subsubsection{1-RDM} % (fold)
% \label{sec:1-RDM}

% subsubsection 1-RDM (end)
% NOTAS SACADAS DEL PERNAL
% mirar tb seccion 1.7.3 del Hergakel
In the coordinate representation of first quantization,
the \textit{first-order reduced density matrix} (1-RDM), $\fordm$, is defined
for an $n$-electron wavefunction, $\Psi$, as
\begin{equation} \label{eq:fordm-coordinate-representation}
    \fordm \left( \mat{x}, \mat{x}\myprime \right) =
    n \idotsint
    \Psi \left( \mat{x}, \mat{x_2}, \ldots,  \mat{x}_n \right)
    \Psi^{*} \left( \mat{x}\myprime, \mat{x_2}, \ldots,  \mat{x}_n \right)
    \dd{ \mat{x}_2} \cdots \dd{ \mat{x}_n}
    ,
\end{equation}
where $ \mat{x} = \left( \mat{r}, s \right)$ is a combined spatial and spin
coordinate.

%TODO: incluir info de seccion 2 del pernal

% An inmidiate advantage of using 1-RDM instead of the electron density, $\rho$,
% is that the kinetic energy is an explicit functional of $\fordm$ but not of
% $\rho$.
Employing the 1-RDM rather than the electron density, $\rho$, has the immediate
advantage of making the kinetic energy an explicit functional of $\fordm$ rather
than $\rho$.

% Then, there is no need to introduce a ficticious noninteracting system.
Then, introducing a fictitious noninteracting system is not necessary.

% Moreover, orbitals present in Reduced Density Matrix Functional Theory (RDMFT)
% are fractionally occupied, so functionals 
% of $\fordm$ seem to be better suited to account for static correlation.
Furthermore, fractionally occupied orbitals are a feature of Reduced Density 
Matrix Functional Theory (RDMFT), which suggests that functionals of $\fordm$ are 
better suited to explain static correlation.

Self-adjointness of $\fordm$, as defined in \cref{eq:fordm-coordinate-representation},
allows for its spectral representation\mycite{lowdin1955quantum}
\begin{equation}
    \fordm \left( \mat{x}, \mat{x}\myprime \right) =
    \sum_{p} n_p 
    \varphi_p \left( \mat{x} \right)
    \varphi_p^{*} \left( \mat{x}\myprime \right)
    .
\end{equation}

The eigenvalues of 1-RDM are called natural occupation numbers,
$\left\{ n_p \right\}$, and the eigenfunctions are known as natural
spin-orbitals, $\left\{ \varphi_p \right\}$.
% For convention, the indices $p,q,r,s$ are referred to natural spin-orbitals
% and $a,b,c,d$ to arbitrary one-eletron functions.
% Also, atomic units are employed throughout the document.
By convention, natural spin-orbitals are denoted by the indices $p$, $q$, $r$, and $s$, 
and arbitrary one-eletron functions by the indices $a$, $b$, $c$, and $d$. 

% Some properties of the natural spin-orbitals and occupation numbers, known as the 
% $n$-representability conditions, are:
% the ensemble n-representability conditions for the first-order reduced density
% matrix have a particularly simple form\mycite{coleman1963structure}
The ensemble $n$-representability conditions, which are some properties of the 
natural spin-orbitals and occupation numbers, have a particularly simple 
form\mycite{coleman1963structure} and read:
\begin{enumerate}
    \item Self-adjointness of $\fordm$: implies orthonormality of the natural orbitals 
        \begin{equation} \label{ec:self-adjoint-condition}
            \int
            \varphi_p^{*} \left( \mat{x} \right)
            \varphi_q \left( \mat{x} \right)
            = \delta_{pq}
            \quad \forall_{p,q} 
            .
        \end{equation}
        
    \item $\fordm$ is assumed to be normalized to a number of electrons, $n$.
        Therefore, the natural occupancies sum up to $n$  
        \begin{equation} \label{ec:normalization-condition}
            \sum_{p} n_p = n
            .
        \end{equation}

    \item Occupation number, $n_p$, is nonnegative and not grater than 1 
    \begin{equation} \label{ec:occ-number-condition}
        0 \leq n_p \leq 1 \quad \forall_{p}
        .
    \end{equation}
\end{enumerate}

% As demonstrated by Gilbert, who extended the Hohenberg-Kohn theorems to nonlocal
% potentials\mycite{gilbert1975hohenbergkohn, donnelly1978elementary}, there
% exists a 1-RDM functional\mycite{gilbert1975hohenbergkohn, hohenberg1964inhomogeneousa}
% such that
Gilbert proved that there is a 1-RDM functional\mycite{gilbert1975hohenbergkohn, hohenberg1964inhomogeneousa}
such that the Hohenberg-Kohn theorems extend to 
nonlocal potentials\mycite{gilbert1975hohenbergkohn, donnelly1978elementary}
\begin{equation} \label{ec:HK-functional}
    \func{E_{\nu}^{\text{HK}}}{\fordm} = 
    \mytrace{\hat{h} \hat{\fordm}} + 
    \Braket{\func{\Psi}{\fordm} | \hat{V}_{\text{ee}} | \func{\Psi}{\fordm}}
    ,
\end{equation}
% where $\func{\Psi}{\fordm}$ denotes a ground state wavefunction pertinent to a
% $\nu$-representable $\fordm$.
% $\hat{h}$ stands for a one-electron Hamiltonian comprising the kinetic
% energy, $\hat{T}$, and external potential, $\hat{V}_{\text{ext}}$ 
where a ground state wavefunction relevant to a $\nu$-representable $\fordm$ 
is denoted by the symbol $\func{\Psi}{\fordm}$.
The one-electron Hamiltonian $\hat{h}$ is composed of up of the external
potential $\hat{V}_{\text{text}}$ and the kinetic energy $\hat{T}$
\begin{equation}
    \hat{h} = \hat{T} + \hat{V}_{\text{ext}}
        ,
\end{equation}
and $\hat{V}_{\text{ee}}$ is an electron interaction operator 
\begin{equation}
    \hat{V}_{\text{ee}} = 
    \sum_{i>j}^{n} \frac{1}{r_{ij}}
    .
\end{equation}

This functional follows the variational principle 
\begin{equation} \label{ec:var-princ-nu-rep}
    \func{E_{\nu}}{\fordm} \ge E_0 \quad \forall_{\fordm \in \text{$\nu$-rep}}
    ,
\end{equation}
where $\nu$-rep denotes a set of pure-state $\nu$-representable 1-RDMs.

% The domain of a density matrix functional to all pure-state $n$-representable
% 1-RDMs was extended by Levy by defining the electron repulsion functional\mycite{doi:10.1073/pnas.76.12.6062, levy1987correlation} 
Levy defined the electron repulsion functional\mycite{doi:10.1073/pnas.76.12.6062, levy1987correlation}, extending the domain of a density matrix 
functional to all pure-state $n$-representable 1-RDMs
\begin{equation} \label{ec:levy-functional}
    \func{E_{\text{ee}}^{\text{L}}}{\fordm} = 
    \min_{\Psi \to \fordm} \Braket{\Psi | \hat{V}_{\text{ee}} | \Psi}
    ,
\end{equation}
and further extended to ensemble $n$-representable 1-RDMs (belonging to a set
``$n$-rep'') by Valone\mycite{valone2008consequences, valone2008one}, so the
exact functional reads 
\begin{equation} \label{ec:ee-functional}
    \func{E_{\text{ee}}}{\fordm} =
    \min_{\sordm^{\left( n \right)}\to \fordm} \mytrace{\hat{ \mat{H}} \hat{\sordm}^{\left( n \right)}}
    ,
\end{equation}
where the minimization is carried out with respect to $n$-electron density
matrices $\sordm^{\left( n \right)}$ that yield $\fordm$.

% For a given external potential, $\hat{V}_{\text{ext}}$, minima of the Hohenberg-Kohn
% functional given in \cref{ec:HK-functional}, the Levy functional given in
% \cref{ec:levy-functional} and that in \cref{ec:ee-functional} defined, respectively,
% for $\nu$-rep, pure-state $n$-representable, and ensemble $n$-representable
% ($n$-rep) 1-RDMs, coincide\mycite{valone2008consequences, nguyen-dang1985variation}.
% Therefore, taking into account a variational principle given in \cref{ec:var-princ-nu-rep},
% one concludes that a functional defined for $n$-rep 1-RDMs yields a ground state
% energy at minimum
The minima of the Levy functional given in \cref{ec:levy-functional}, the 
Hohenberg-Kohn functional given in \cref{ec:HK-functional}, and that defined in
\cref{ec:ee-functional}, respectively for $\nu$-rep, pure-state $n$-representable,
and ensemble $n$-representable ($n$-rep) 1-RDMs coincide\mycite{valone2008consequences, nguyen-dang1985variation}
for a given external potential, $\hat{V}_{\text{ext}}$.
Thus, considering the variational principle given in \cref{ec:var-princ-nu-rep}, 
it can be concluded that a ground state energy at minimum is produced by a 
functional defined for $n$-rep 1-RDMs
\begin{equation} \label{ec:gs-energy}
    E_0 = \min_{\fordm \in \text{$n$-rep}}
    \left\{ \mytrace{\hat{h} \hat{\fordm}} + \func{E_{\text{ee}}}{\fordm} \right\}
    .
\end{equation}

Indeed, \cref{ec:self-adjoint-condition,ec:normalization-condition,ec:occ-number-condition,ec:gs-energy}
are the foundation for RDMFT.

There are two cases where the exact forms of $\func{E_{\text{ee}}}{\fordm}$ are known:
\begin{enumerate}
    \item \textbf{$n$-electron noninteracting systems}: the 1-RDM corresponding
        to a single determinant wavefunction is idempotent, which implies integer
        (0 or 1) values of the natural occupation numbers 
        \begin{equation}
            \hat{\fordm}^{2} = \hat{\fordm}
            \Leftrightarrow
            n_p = 0  \vee n_p = 1
            \quad \forall_{p}
            .
        \end{equation}

        In this case, a two-electron reduced density matrix (2-RDM), $\sordm$, is defined
        for a general wavefunction, $\Psi$, as 
        \begin{equation}
            \sordm \left( \mat{x}_1, \mat{x}_2, \mat{x}_1\myprime, \mat{x}_2\myprime \right)
            =
            n\left( n - 1 \right) \idotsint
            \Psi \left( \mat{x}_1, \mat{x}_2, \mat{x}_3, \ldots,  \mat{x}_n \right)
            \Psi^{*} \left( \mat{x}_1\myprime, \mat{x}_2\myprime, \mat{x}_3, \ldots,  \mat{x}_n \right)
            \dd{ \mat{x}_3} \cdots \dd{ \mat{x}_n}
            ,
        \end{equation}
        which is explicitly expressible in terms of 1-RDM if the wavefunction takes
        the form of a Slater determinant 
        \begin{equation} \label{eq:HF-approximation-sordm-1}
            \sordm \left( \mat{x}_1, \mat{x}_2, \mat{x}_1\myprime, \mat{x}_2\myprime \right)
            = 
            \fordm \left( \mat{x}_1, \mat{x}_1\myprime \right)
            \fordm \left( \mat{x}_2, \mat{x}_2\myprime \right)
            -
            \fordm \left( \mat{x}_1, \mat{x}_2\myprime \right)
            \fordm \left( \mat{x}_2, \mat{x}_1\myprime \right)
            .
        \end{equation}

        The electron interaction functional corresponding to such a
        noninteracting 2-RDM reads 
        \begin{equation}
            \func{E_{\text{ee}}^{\text{HF}}}{\fordm} =
            \func{E_{\text{H}}}{\fordm} + \func{E_{\text{x}}}{\fordm}
            ,
        \end{equation}
        % refered to as the Hartree-Fock functional, because the optimization of
        % the functional leads to an idempotent density matrix coinciding with
        % the solution to the HF equations\mycite{lieb1981variational}.
        referred to as the Hartree-Fock functional as the HF equations' 
        solution coincides with an idempotent density matrix that results 
        from functional optimization\mycite{lieb1981variational}.
        The Hartree functional, $E_{\text{H}}$, describes the classical
        part of electron interaction 
        \begin{equation}
            \func{E_{\text{H}}}{\fordm} = \frac{1}{2} \iint
            \frac{
                \fordm \left( \mat{x}, \mat{x} \right)
                \fordm \left( \mat{x}\myprime, \mat{x}\myprime \right)
            }{\left| \mat{r} - \mat{r}\myprime \right|}
            \dd{ \mat{x}} \dd{ \mat{x}\myprime}
            ,
        \end{equation}
        whereas the exchange functional, $E_{\text{x}}$, reads 
        \begin{equation}
            \func{E_{\text{x}}}{\fordm} = -\frac{1}{2} \iint
            \frac{
                \fordm \left( \mat{x}, \mat{x}\myprime \right)
                \fordm \left( \mat{x}\myprime, \mat{x} \right)
            }{\left| \mat{r} - \mat{r}\myprime \right|}
            \dd{ \mat{x}} \dd{ \mat{x}\myprime}
            .
        \end{equation}

    \item \textbf{Two-electron closed-shell system}: an interacting two-electron
        species.

        % (see sec 2.1 of Pernal)
        The exact density matrix functional for a two-electron system is
        known\mycite{goedecker1998natural, kutzelnigg1963loesung}.

        Based on the work of Löwdin and Shull (LS)\mycite{shull1956correlation},
        a Slater-determinant-expansion of a singlet wavefunction (assumed to be
        real-valued) in the natural spin-orbitals basis $\left\{ \varphi_p \right\}$ 
        is entirely given by \textit{diagonal} determinants composed of the
        spin-orbitals that share spatial parts 
        \begin{equation}
            \psi^{\text{LS}} =
            \sum_{p} c_p \left| \varphi_p \varphi_{\bar{p}} \right|
            ,
        \end{equation}
        % where $p$ and $\bar{p}$ are spin-orbitals of the opposite spin and
        % $\left| \varphi_p \varphi_{\bar{p}} \right|$ denotes a normalized
        % Slater determinant.
        where $\left| \varphi_p \varphi_{\bar{p}} \right|$ indicates a 
        normalized Slater determinant and $p$ and $\bar{p}$ are
        spin-orbitals of opposite spin.
        The normalization of $\psi^{\text{LS}}$ imposes
        that $\displaystyle\sum_{p} c_p^{2} = 1$.

        % The 1-RDM given by this wavefunction is directly in its spectral
        % representation, which indicates that squares of the expansion
        % coefficients are simply the natural occupation numbers 
        This wavefunction provides the 1-RDM directly in its spectral 
        representation, implying that the squares of the expansion 
        coefficients are just the natural occupation numbers
        \begin{equation}
            n_p = c_p^2\quad \forall_p
            .
        \end{equation}

        The energy is given by 
        \begin{equation}
            E = \Braket{\psi^{\text{LS}} | \hat{H} | \psi^{\text{LS}}} =
            \sum_{p} c_p^2 h_{pp} + \frac{1}{2} \sum_{pq} 
            c_p c_q \Braket{pp | qq}
            ,
        \end{equation}
        assuming that the coefficients corresponding to spin-orbitals of opposite
        spins and same spatial parts are equal.

        By minimizing the energy with respecto to $\left\{ c_p \right\}$ and
        $\left\{ \varphi_p \right\}$, an exact electron interaction density
        matrix functional can be immediately written as 
        \begin{equation} \label{eq:LS-functional}
            E_{\text{ee}}^{\text{LS}} \left[ \fordm \right] =
            \frac{1}{2} \min_{ \left\{ f_q \right\}} \sum_{pq}
            f_p f_q \sqrt{n_p n_q} \Braket{pq | qp}
            ,
        \end{equation}
        with 
        \begin{equation}
            f_p = \pm 1\quad \forall_p
            ,
        \end{equation}
        and $ \Braket{pp | qq} = \Braket{pq | qp}$, as the orbitals are real.

        % It is known that for two-electron atoms and molecules at equilibrium
        % geometry the sign of the factor $f_1$ corresponding to the highest
        % occupation $n_1$ is predominantly opposite to signs
        % $\left\{ f_p \right\}$ of all other factors corresponding to weakly
        % occupied $\left( n_p < \frac{1}{2} \right)$ orbitals\mycite{goedecker1998natural}.
        For two-electron atoms and molecules in equilibrium geometry, it is 
        known that the sign of the factor $f_1$ corresponding to the highest 
        occupation $n_1$ is predominantly opposite to the signs of all other 
        factors corresponding to weakly occupied $\left( n_p < \frac{1}{2} \right)$
        orbitals\mycite{goedecker1998natural}.

        % There are known cases when this rule is violated\mycite{goedecker1998natural, sheng2013natural, giesbertz2013longrange, cioslowski2000ground},
        % in which a two-electron functional explicitly depending on the occupation
        % noumbers is defined as 
        There are known cases when this condition is violated\mycite{goedecker1998natural, sheng2013natural, giesbertz2013longrange, 
        cioslowski2000ground}, in which a two-electron functional that depends 
        explicitly on the occupation numbers is defined as
        \begin{equation}
            \tilde{E}_{\text{ee}}^{\text{LS}} \left[ \fordm \right] =
            \frac{1}{2} \sum_{pq}  G_{pq}^{\text{LS}} \Braket{pq | qp},
        \end{equation}
        with 
        \begin{equation}
            G_{pq}^{\text{LS}} =
            \begin{cases}
                n_p & p = q \\
                - \sqrt{n_p n_q} & p = 1,\ q > 1\ \mathrm{or}\ p>1,\ q=1 \\
                \sqrt{n_p n_q} & \text{otherwise}
            \end{cases}
            ,
        \end{equation}
        which, although not always fully equivalent to the exact LS functional
        given in \cref{eq:LS-functional}, is a good approximation.

        Other functionals like the correction done to the Müller\mycite{muller1984explicit}
        or, as refered in this text, Buijse and Baerends functional\mycite{buijse2002approximate, buijse1991electron},
        denoted as the BB-corrected (BBC) functionals\mycite{gritsenko2005improved}
        have been developed.

        % Also, the LS functional was extended for the more general closed-shell
        % $n$ (even) electron ansatz involving all determinants arising from
        % diagonal double, diagonal quadruple, etc. excitations.
        % %
        % The extended Löwdin-Shull (ELS) functional\mycite{pernal2004phase, mentel2014density}
        % is applicable to systems for which a set of
        % the natural spin-orbitals can be partitioned into \textit{inner} orbitals
        % localized on atoms and the occupancies close to 1, and \textit{outer}
        % orbitals including a bonding orbital and all weakly occupied orbitals.
        % % eq 44 y 45 del Pernal
        % It has been designed to treat only molecules with one single bond and no 
        % lone pairs, aiming at providing a balanced description of the dynamic
        % and static correlation.
        Additionally, the LS functional has been extended to include all 
        determinants resulting from diagonal double, diagonal quadruple, etc. 
        excitations in the more general closed-shell $n$ (even) electron ansatz.
        %
        For systems for which a set of the natural spin-orbitals can be divided 
        into \textit{inner} orbitals, which are localized on atoms and have 
        occupancies close to 1, and \textit{outer} orbitals, which include a 
        bonding orbital and all weakly occupied orbitals, the extended Löwdin-
        Shull (ELS) functional\mycite{pernal2004phase, mentel2014density} is 
        applicable.
        %eq 44 and 45 of the Pernal
        In order to provide a balanced description of the dynamic and static 
        correlation, it has been developed to treat only molecules with a 
        single bond and no lone pairs.

\end{enumerate}

Therefore, exact density matrix functionals for an uncorrelated and a strongly
correlated electron pair are known.

\subsubsection{2-RDM} % (fold)
\label{sec:2-RDM}

In the coordinate representation, the \textit{second-order reduced density
matrix} (2-RDM), $\sordm$, is defined as 
\begin{equation} \label{eq:sordm-coordinate-representation}
    \sordm \left( \mat{x}_1, \mat{x}_2, \mat{x}_1\myprime, \mat{x}_2\myprime \right)
    =
    \frac{n\left( n - 1 \right)}{2} \idotsint
    \Psi \left( \mat{x}_1, \mat{x}_2, \mat{x}_3, \ldots,  \mat{x}_n \right)
    \Psi^{*} \left( \mat{x}_1\myprime, \mat{x}_2\myprime, \mat{x}_3, \ldots,  \mat{x}_n \right)
    \dd{ \mat{x}_3} \cdots \dd{ \mat{x}_n}
    .
\end{equation}

% indices a,b,c,d:  arbitrary 1-e functions
% Owing to the two-particle character of the electron interaction, the electronic
% repulsion energy $E_{\text{ee}}$, given a system defined by the ground state
% wavefunction $ \Ket{0}$, results from contraction of the two-electron
% reduced density matrix components, $\sordm_{abcd}$, with two-electron integrals
% $ \Braket{ab | cd}$ 
Due to the two-particle nature of the electron interaction, the contraction of 
the two-electron reduced density matrix components, $\sordm_{abcd}$, with two-
electron integrals $ \Braket{ab | cd}$, yields the electronic repulsion energy 
$E_{\text{ee}}$ given a system defined by the ground state wavefunction $ \Ket{0}$
\begin{equation} \label{eq:Eee-as-function-of-2RDM}
    E_{\text{ee}}
    =
    \frac{1}{2}
    \sum_{abcd} \sordm_{abcd} \Braket{cd | ab}
    ,
\end{equation}
where the 2-RDM, for a given basis set $\left\{ \chi_a \right\}$, is defined as 
\begin{equation}
    \sordm_{abcd} =
    \Braket{0 | \hat{c}^{\dagger} \hat{d}^{\dagger} \hat{b} \hat{a} | 0}
    .
\end{equation}

Formally, the 2-RDM is a functional of the 1-RDM. %TODO: incluir alguna cita

% Recalling the Hartree-Fock approximation given in \cref{eq:HF-approximation-sordm-1},
% in which $\sordm$ in the representation of the natural spin-orbitals is given
% solely in terms of the occupation numbers 
Using the Hartree-Fock approximation stated in \cref{eq:HF-approximation-sordm-1},
where $\sordm$ in the representation of the natural spin-orbitals is given 
solely in terms of the occupation numbers
\begin{equation}
    \sordm_{pqrs}^{\text{HF}}
    =
    n_p n_q \left( \delta_{pr} \delta_{qs} - \delta_{ps} \delta_{qr} \right)
    ,
\end{equation}
% and making the assumption that the elements of $\sordm$ are functions of the natural occupation
% numbers, $\left\{ n_p \right\}$, and the basis set is composed of natural 
% spin-orbitals, $\left\{ \varphi_p \right\}$, the whole dependence of $E_{\text{ee}}$ 
% on $\varphi_p$ is included in two-electron integrals.
and assuming that the elements of $\sordm$ are functions of the natural 
occupation numbers, $\left\{ n_p \right\}$, and the basis set is formed by 
natural spin-orbitals, $\left\{ \varphi_p \right\}$, the entire dependence of
$E_{\text{ee}}$ on $\varphi_p$ is included in two-electron integrals.

% Then, the electron-electron interaction functional given in \cref{ec:ee-functional}
% can be approximaded with the exploit given in \cref{eq:Eee-as-function-of-2RDM}
% as
Then, using the exploit provided in \cref{eq:Eee-as-function-of-2RDM}, the 
electron-electron interaction functional given in \cref{ec:ee-functional} may 
be approximated as
\begin{equation}
    E_{\text{ee}} \left[ \fordm \right]
    =
    \frac{1}{2} \sum_{pqrs}
    \sordm_{pqrs}\left[ \left\{ n_t \right\} \right] 
    \Braket{rs | pq}
    .
\end{equation}

In most approximations proposed so far, $E_{\text{ee}} \left[ \fordm \right]$
is an explicit function of the occupation numbers and the natural spin-orbitals. 

% ------------------------------
\subsection{Properties of density matrices} % (fold)
\label{sec:properties-of-density-matrices}
% ------------------------------
% The electronic energy can be written as 
% \begin{equation}
%     E = 
%     \sum_{\sigma}  \sum_{pq} 
%     \fordm_{pq}^{\sigma \sigma} h_{q \sigma,p \sigma}
%     +
%     \sum_{\sigma \sigma^{\prime}} \sum_{pqrs}
%     \sordm_{pqrs}^{\sigma \sigma^{\prime} \sigma \sigma^{\prime}}
%     \Braket{\psi_{r \sigma} \psi_{s \sigma^{\prime}} | \psi_{p \sigma} \psi_{q \sigma^{\prime}}}
%     ,
% \end{equation}
% with 
% \begin{align}
%     h_{pq} 
%     &=
%     \Braket{\psi_p | -\frac{1}{2} \Delta + V\left( \mat{r} \right) | \psi_q}
%     ,
%     \\
%     \Braket{\psi_p \psi_q | \psi_r \psi_s}
%     &=
%     \int \psi_p^{*}\left( \mat{r}_1 \right) \psi_q^{*} \left( \mat{r}_2 \right)  
%     \frac{e^{2}}{\left| \mat{r}_1 - \mat{r}_2 \right|}
%     \psi_r\left( \mat{r}_1 \right) \psi_s\left( \mat{r}_2 \right)
%     \dd^{3}\mat{r}_1 \dd^{3}\mat{r}_2
%     .
% \end{align}

In the second quantization formalism, the electronic molecular Hamiltonian in 
the spin-orbital basis is given by
\begin{equation}
    \hat{H} =
    \sum_{pq} h_{pq} a_p^{\dagger} a_q
    +
    \frac{1}{2} \sum_{pqrs} \PBraket{pr | qs} a_p^{\dagger} a_q^{\dagger} a_s a_r
    .
\end{equation}

Using restricted spin-orbitals, it is convenient to define the operators 
\begin{equation}
    E_{pq} =
    \sum_{\sigma} a_{p\sigma}^{\dagger} a_{q\sigma}
    ,
\end{equation}
such that 
\begin{equation}
    \left[ E_{pq}, E_{rs} \right] =
    \delta_{qr} E_{ps} - \delta_{ps} E_{qr}
    .
\end{equation}

Then, the Hamiltonian can be written as 
\begin{equation}
    \hat{H} =
    \sum_{pq} h_{pq} E_{pq}
    + \frac{1}{2}
    \sum_{pqrs} \PBraket{pq | rs}\left( E_{pq} E_{rs} - \delta_{qr}E_{ps} \right)
    .
\end{equation}

% The expectation value of $\hat{H}$ with respect to a normalized reference state
% $\Ket{0}$ written as a linear combination of ON vectors 
Given a normalized reference state
$\Ket{0}$ written as a linear combination of ON vectors 
\begin{equation}
    \Ket{0} = \sum_{k} c_k \Ket{k}
    , \quad
    \Braket{0 | 0} = 1
    ,
\end{equation}
the expectation value of $\hat{H}$ with respect to $\Ket{0}$ reads
\begin{equation} \label{eq:GS-energy-1}
    E_0 = \Braket{0 | \hat{H} | 0} =
    \sum_{pq} h_{pq} D_{pq} + \frac{1}{2} \sum_{pqrs}
    \PBraket{pq | rs} \left( d_{pqrs} - \delta_{qr} D_{ps} \right)
    ,
\end{equation}
where 
\begin{equation}
    d_{pqrs} = \Braket{0 | E_{pq} E_{rs} | 0}
    ,
\end{equation}
and
\begin{equation} \label{eq:oedm-spin-representation}
    D_{pq} = \Braket{0 | E_{pq} | 0} \equiv \oedm_{pq}
    ,
\end{equation}
are the elements (densities) of an $m \times m$ Hermitian matrix, the
\textit{one-electron spin-orbital density matrix}, $ \mat{\oedm}$, since 
\begin{equation}
    D_{pq}^{*} =
    \Braket{0 | a_p^{\dagger} a_q | 0} =
    \Braket{0 | a_q^{\dagger} a_p | 0} =
    D_{qp}
    ,
\end{equation}
which is also symmetric for real wave functions.

% The one-electron density matrix is positive semidefinite since its elements are 
% either trivially equal to zero or inner products of states in the subspace
% $F\left( m, n-1 \right)$.
% The diagonal elements of $ \mat{\oedm}$ are the expectation values of the
% occupation-number operators, $n_p$, in $F\left( m,n \right)$, and are referred to as
% the occupation numbers, $\omega_p$, of the electronic state 
Since the elements of the one-electron density matrix are inner products of 
states in the subspace $F\left( m, n-1 \right)$ or trivially equal to zero, 
the one-electron density matrix is positive semidefinite.
The occupation numbers, $\omega_p$, of the electronic state are given by the
diagonal elements of $ \mat{\oedm}$, i.e. the expectation values of the 
occupation-number operators, $n_p$, in $F\left( m,n \right)$
\begin{equation}
    \omega_p = \oedm_{pp} = \Braket{0 | n_p | 0}
    .
\end{equation}

The diagonal elements of $ \mat{\oedm}$ reduce to the usual occupation numbers,
$k_p$, whenever the reference state is an eigenfunction of the ON operators, i.e.
when the reference state is an ON vector 
\begin{equation}
    \Braket{k | n_p | k} = k_p
    .
\end{equation}

Considering that the ON operators are projectors, $\omega_p$ can be written 
over the projected electronic state 
\begin{equation}
    n_p \Ket{0} = \sum_{k} k_p c_k \Ket{k}
    ,
\end{equation}
as 
the squared norm of the part of the reference state where the spin orbital
$\varphi_p$ is occupied in each ON vector
\begin{equation}
    \omega_p =
    \Braket{0 | n_p n_p | 0} =
    \sum_{k} k_p \left| c_k \right|^2
    .
\end{equation}

Recalling the $n$-representability conditions 
(\cref{ec:occ-number-condition, ec:normalization-condition}),
the occupation numbers are real numbers between zero and one 
and its sum, the trace of the density matrix, is equal to the total number
of electrons in the system 
\begin{equation}
    \trace \mat{\oedm} =
    \sum_{p} \omega_p =
    \sum_{p} \Braket{0 | n_p | 0} =
    n
    .
\end{equation}

Since $ \mat{\oedm}$ is Hermitian, it can be diagonalized by the spectral
decomposition for a unitary matrix, $ \mat{U}$, as
\begin{equation} \label{eq:diagonalization-oedm}
    \mat{\oedm} =
    \mat{U} \lambda \mat{U}^{\dagger}
    .
\end{equation}

The eigenvalues, $\lambda_p$, are real numbers 
\begin{equation}
    0 \le \lambda_p \le 1
    ,
\end{equation}
known as the natural-orbital occupation numbers, which also fulfill 
\begin{equation}
    \sum_{p} \lambda_p = n
    .
\end{equation}

A new set of spin orbitals, the natural spin orbitals, are obtained from the
eigenvectors, the columns of $ \mat{U}$, resulted from 
\cref{eq:diagonalization-oedm}.

With the definition of the one-electron density matrix given in 
\cref{eq:oedm-spin-representation}, \cref{eq:GS-energy-1} can be further
simplified introducing the \textit{two-electron spin-orbital density matrix}
\begin{equation} \label{eq:tedm-spin-representation}
    \tedm_{pqrs} = d_{pqrs} - \delta_{qr} D_{ps}
    ,
\end{equation}
as
\begin{equation} \label{eq:GS-energy-2}
    E_0 =
    \sum_{pq} h_{pq} \oedm_{pq} + \frac{1}{2} \sum_{pqrs}
    \PBraket{pq | rs} \tedm_{pqrs}
    .
\end{equation}

The elements of $ \mat{\tedm}$ are not all independent because of the
anticommutation relations between the creation and annihilation operators 
\begin{equation}
    \tedm_{pqrs} = - \tedm_{rqps} = - \tedm_{ps rq} = \tedm_{rspq}
    .
\end{equation}

Also, in accordance with the Pauli principle 
\begin{equation}
    \tedm_{pqps} = \tedm_{pqrq} = \tedm_{pqpq} = 0
    .
\end{equation}

The two-electron density matrix can be rewritten as a
$m\left( m-1 \right) /2 \times m\left( m-1 \right) /2$ 
matrix, $ \mat{T}$ 
\begin{equation}
    T_{pq,rs} = \Braket{0 | a_p^{\dagger} a_q^{\dagger} a_s a_r | 0}
    , \quad
    p>q,\ r>s
    ,
\end{equation}
with composite indices $pq$, composing a subset of $ \mat{\tedm}$ by a
reordering of the middle indices 
\begin{equation}
    T_{pq,rs} = \tedm_{prqs}
    , \quad
    p>q,\ r>s
    .
\end{equation}

The matrix $ \mat{T}$ is also Hermitian and, therefore, symmetric for real
wave functions.
Furthermore, it is positive semidefinite since its elements are either zero
or inner products of states in $F\left( m, n-2 \right)$.

The diagonal elements of $ \mat{T}$, considering that $p > q$ and introducing
the ON operators, are given by
\begin{equation}
    \omega_{pq} = T_{pq,pq} =
    \Braket{0 | a_p^{\dagger} a_q^{\dagger} a_q a_p | 0} =
    \Braket{0 | n_p n_q | 0},
\end{equation}
which can be interpreted as simultaneous occupations of pairs of spin orbitals
from the part of the wave function where the spin
orbitals $\varphi_p$ and $\varphi_q$ are simultaneously occupied.
It can be simply denoted as pair occupations, and fulfill the following conditions:
\begin{enumerate}
        % Simultaneous occupation of a given spin-orbital pair cannot exceed those of
        % the individual spin orbitals 
    \item The simultaneous occupation of a given spin-orbital cannot be greater than that
        of the individual spin orbitals
        \begin{equation}
            0 \le \omega_{pq} \le \min\left[ \omega_p, \omega_q \right] \le 1
            .
        \end{equation}

    \item The sum of all pair occupations $\omega_{pq}$ is equal to the number of
        electron pairs in the system 
        \begin{equation}
            \trace \mat{T} =
            \sum_{p>q} \Braket{0 | n_p n_q | 0} =
            \frac{1}{2} \sum_{pq} \Braket{0 | n_p n_q | 0}
            - \frac{1}{2} \sum_{p} \Braket{0 | n_p | 0} =
            \frac{1}{2} n \left( n - 1 \right)
            .
        \end{equation}
\end{enumerate}

% For a state containing a single ON vector, $\Ket{k}$, the $ \mat{T}$ matrix
% has a simple diagonal structure and may be constructed directly from the
% one-electron density matrix as
The $ \mat{T}$ matrix for a state with a single ON vector, $\Ket{k}$, has a 
straightforward diagonal form and may be constructed directly from the one-
electron density matrix as
\begin{equation}
    T_{pq,rs}^{\Ket{k}} = D_{pq}^{\Ket{k}} D_{qs}^{\Ket{k}}
    .
\end{equation}
% and, likewise, the expectation value of any one- and two-electron operator
% may be obtained directly from the one-electron density matrix.
Similarly, the one-electron density matrix can be used to obtain the 
expectation value of any one or two-electron operator.

% This observation is consistent with the picture of an uncorrelated description
% of the electronic system where the simultaneoys occupations of pairs of spin
% orbitals are just the products of the individual occupations.
% For a general electronic state containing more than one ON vector , $ \mat{T}$ 
% is in general not diagonal and cannot be generated directly from the one-electron
% density elements. In practice, it can be diagonalized with an unitary matrix
% as shown for the one-electron density matrix.
The picture of an uncorrelated description of the electronic system, in which 
the simultaneous occupations of pairs of spin orbitals are merely the products 
of the individual occupations, is compatible with this result.
In the case of a general electronic state containing several ON vectors, $\mat{T}$
is generally non-diagonal and cannot be produced directly from the elements of 
the one electron density matrix. As demonstrated for the one-electron density 
matrix in \cref{eq:diagonalization-oedm}, $ \mat{\tedm}$ can be diagonalized
in practice using the spectral decomposition.

Therefore, it can be stated that the one-electron density matrix probes the
individual occupancies of the spin orbitals and describes how the $n$ electrons
are distributed among the $m$ spin orbitals, whereas the two-electron density
matrix probes the simultaneous occupations of the spin orbitals and describes
how the $n\left( n-1 \right) /2$ electron pairs are distributed among the 
$m\left( m-1 \right) /2$ spin-orbital pairs\mycite{helgaker2000second}.

% ------------
\subsubsection{Equivalence between spin-orbital and coordinate representations}
% ------------
In this section, the equivalence relations between first and second quantization
formalisms for the density matrices are established.

% A relationship between the definition of the 1-RDM
% (\cref{eq:fordm-coordinate-representation})
% and 2-RDM (\cref{eq:sordm-coordinate-representation}) in the first quantization
% formalism, in the coordinate representation, and the one and two-electron
% density matrices in the spin-orbital representation 
% (\cref{eq:oedm-spin-representation,eq:tedm-spin-representation})
% can be stablished.

The one-electron density matrix in the spin-orbital representation was
introduced in second quantization for the evaluation of the expectation values
of one-electron operators in the form 
\begin{equation}
    \Braket{0 | \hat{g} | 0} =
    \sum_{pq} \oedm_{pq} g_{pq}
    ,
\end{equation}
with the $g_{pq}$ integrals 
\begin{equation}
    g_{pq} = \int
    \varphi_p \left( \mat{x}_1 \right)^{*} 
    g \left( \mat{x}_1 \right) 
    \varphi_q \left( \mat{x}_1 \right)
    \dd{ \mat{x}_1}
    .
\end{equation}

Combining both, the 1-RDM in the coordinates representation given in 
\cref{eq:fordm-coordinate-representation} can be written
in terms of the one-electron density matrix in the spin-orbital representation
given in \cref{eq:oedm-spin-representation}, as
\begin{equation}
    \fordm \left( \mat{x}_1, \mat{x}_1^{\prime} \right) =
    \sum_{pq} \oedm_{pq} 
    \varphi_p^{*} \left( \mat{x}_1^{\prime} \right)
    \varphi_q \left( \mat{x}_1 \right)
    .
\end{equation}

Similarly, the following relationship between the two-electron density matrix
in the coordinate representation (\cref{eq:sordm-coordinate-representation}) 
and that in the spin-orbital representation (\cref{eq:tedm-spin-representation})
may be established
\begin{equation}
    \sordm \left( \mat{x}_1, \mat{x}_2, \mat{x}_1^{\prime}, \mat{x}_2^{\prime} \right)
    =
    \frac{1}{2} \sum_{pqrs} \tedm_{pqrs}
    \varphi_p^{*}\left( \mat{x}_1^{\prime} \right)
    \varphi_q    \left( \mat{x}_1          \right)
    \varphi_r^{*}\left( \mat{x}_2^{\prime} \right)
    \varphi_s    \left( \mat{x}_2          \right)
    .
\end{equation}


% subsubsection Energy (end)

% ---------
\subsection{Construction of 2-RDM in terms of 1-RDM}
% seccion 2.2 Pernal
% ---------

% There is a variety of
% conditions to be imposed on the 2-RDM which may aid in its
% construction\mycite{cioslowski2002systematic}

Assuming the cumulant expansion of 2-RDM\mycite{staroverov2002optimization} which
consists of writing $\sordm$ as the antisymmetrized product of $\fordm$ and
the cumulant part, $\lambda$, being a functional of $\fordm$ 
\begin{equation}
    \sordm_{pqrs} =
    n_p n_q \left( \delta_{pr} \delta_{qs} - \delta_{ps} \delta_{qr} \right)
    + \lambda_{pqrs} \left[ \fordm \right]
    ,
\end{equation}
novel one-electron density matrix functionals can be developed by finding
approximations for the cumulant part imposing known conditions which the exact
cumulant satisfies.

\textcolor{red}{(\ldots)}

%% Functionals PNOFi by Piris
% Reconstructionf the 2-RDM in terms of 1-RDM implies the following equality
% conditions satisfied by the $n$-representable 2-RDM are imposed
% \begin{enumerate}
%     \item Hermicity 
%     \begin{equation}
%         \sordm_{pqrs} = \sordm_{rspq}^{*},
%     \end{equation}
%     %
%     \item antisymmetry 
%     \begin{equation}
%         \sordm_{pqrs} = -\sordm_{qprs} = -\sordm_{pqsr},
%     \end{equation}
%     %
%     \item the sum rule 
%     \begin{equation}
%         \sum_{q} \sordm_{pqrq} = \left( n - 1 \right) n_p \delta_{pr},
%     \end{equation}
% \end{enumerate}
% ---------
\subsection{Approximate functionals}
% ---------
% Pag 130 del Pernal
In the development of approximate functionals, it is convenient to search for
approximations to the exchange-correlation term, $E_{\text{xc}}$, defined as
\begin{equation} \label{ec:xc-functional}
    \func{E_{\text{xc}}}{\fordm} =
    \func{E_{\text{ee}}}{\fordm} - \func{E_{\text{H}}}{\fordm}
    ,
\end{equation}
which can be decomposed into an exchange part (given in \cref{ec:xc-functional})
and the correlation functional, $E_{\text{c}}$ 
\begin{equation}
    \func{E_{\text{c}}}{\fordm} =
    \func{E_{\text{xc}}}{\fordm} - \func{E_{\text{x}}}{\fordm}
    .
\end{equation}

\textcolor{red}{(\ldots)}

% Summary & outlook sec 6 Pernal
