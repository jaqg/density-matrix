\graphicspath{{./figures/}}

% ------
\section{Introduction}
% ------
Second quantization formalism is used in the document. 
Consequently, some useful definitions\mycite{helgaker2000second} are recalled
in \cref{sec:second-quantization}.

Also, Dirac (braket) notation is mainly used in the document except for 
\cref{sec:methodology-calculations}, where the Mulliken (chemistry) notation
is used.

The two-electron integrals are given, in Dirac notation, by 
\begin{equation} \label{eq:dirac-notation}
    \Braket{pq | rs} =
    \iint 
    \psi_p^{*} \left( \mat{r}_1 \right)
    \psi_q^{*} \left( \mat{r}_2 \right)
    \frac{1}{\left| \mat{r}_1 - \mat{r}_2 \right|}
    \psi_r \left( \mat{r}_1 \right)
    \psi_s \left( \mat{r}_2 \right)
    \dd{ \mat{r}_1} \dd{ \mat{r}_2}
    ,
\end{equation}
and in Mulliken notation by
\begin{equation} \label{eq:mulliken-notation}
    \PBraket{pq | rs} =
    \iint 
    \psi_p^{*} \left( \mat{r}_1 \right)
    \psi_q     \left( \mat{r}_1 \right)
    \frac{1}{\left| \mat{r}_1 - \mat{r}_2 \right|}
    \psi_r^{*} \left( \mat{r}_2 \right)
    \psi_s     \left( \mat{r}_2 \right)
    \dd{ \mat{r}_1} \dd{ \mat{r}_2}
    ,
\end{equation}
with the equivalence 
\begin{equation} \label{eq:dirac-mulliken-equivalence}
    \Braket{pq | rs} = \PBraket{pr | qs}
    .
\end{equation}

Additionally, atomic units are used all throughout the document.
