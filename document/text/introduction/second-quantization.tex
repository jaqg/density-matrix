\subsection{Second quantization} % (fold)
\label{sec:second-quantization}

In second quantization, an abstract linear vector space, the \textit{Fock} space,
is introduced.
Each deerminant in this space is represented by an \textit{occupation-number}
(ON) \textit{vector}, $ \Ket{k}$ 
\begin{equation}
    \Ket{k} =
    \Ket{k_1,k_2,\ldots,k_M}
    ,
\end{equation}
with $k_P = 1$ if $P$ is an occupied orbitals and $k_P = 0$ if $P$ is unoccupied.

Also, the \textit{vacuum state} is defined as a vector in the subspace
$F\left( M,0 \right)$ consisting of ON vectors with no electrons 
\begin{equation}
    \Ket{\text{vac}} =
    \Ket{0_1, 0_2, \ldots, 0_M}
    , \quad
    \Braket{\text{vac} | \text{vac}} = 1
    .
\end{equation}

All operators and states can be constructed from a set of elementary
creation and annihilation operators, $\left\{ a^{\dagger} \right\}$, 
$\left\{ a \right\}$.
The creation operators are defined by the relations
\begin{align}
    a_P^{\dagger} \Ket{k_1,k_2,\ldots,0_P,\ldots,k_M} &=
    \Theta_P^{\Ket{k}}
    \Ket{k_1, k_2, \ldots, 1_P, \ldots, k_M}
    , \\
    a_P^{\dagger} \Ket{k_1, k_2, \ldots, 1_P, \ldots, k_M} &= 0
    ,
\end{align}
and the annihilation operators as
\begin{align}
    a_P \Ket{k_1,k_2,\ldots,0_P,\ldots,k_M} &= 0
    , \\
    a_P \Ket{k_1, k_2, \ldots, 1_P, \ldots, k_M} &=
    \Theta_P^{\Ket{k}} \delta_{k_P,1}
    \Ket{k_1, k_2, \ldots, 0_P, \ldots, k_M}
    ,
\end{align}
with the phase factor $\Theta_P^{\Ket{k}}$  
\begin{equation}
    \Theta_P^{\Ket{k}} =
    \prod_{Q=1}^{P-1} \left( -1 \right)^{k_Q}
    ,
\end{equation}
which $\Theta_P^{\Ket{k}}=1$ for an even number of electrons in the spin-orbitals
basis, $Q < P$, and  $\Theta_P^{\Ket{k}}=-1$ for an odd number.

The creation and annihilation operators satisfy the following anticommutation
rules
\begin{align}
    \left[ a_p,q_q \right]_{+}
    &=
    a_p a_q + a_q a_p = 0
    , \\
    \left[ a_p^{\dagger}, a_q^{\dagger} \right]_{+}
    &= a_p^{\dagger} a_q^{\dagger} + a_q^{\dagger} a_p^{\dagger} = 0
    , \\
    \left[ a_q, a_q^{\dagger} \right]_{+}
    &=
    a_p a_q^{\dagger} + a_q^{\dagger} a_p = \delta_{pq},
\end{align}
which guarantees that the Antisymmetry Principle is fulfilled.

For this set of creation and annihilation operators, the ON operators are
defined as
\begin{equation}
    n_p = a_p^{\dagger} a_p
    .
\end{equation}

For a given ON vector $\Ket{k}$, the occupation number $k_p$ is obtained by
counting the number of electrons in spin orbital $p$
\begin{equation}
    n_p \Ket{k} =
    a_p^{\dagger} a_p \Ket{k} =
    k_p \Ket{k}
    .
\end{equation}

The ON operators are Hermitian 
\begin{equation}
    n_p^{\dagger} =
    \left( a_p^{\dagger} a_p \right)^{\dagger} =
    a_p^{\dagger} a_p =
    n_p
    ,
\end{equation}
commute among themselves 
\begin{equation}
    n_p n_q \Ket{k} =
    k_p k_q \Ket{k} =
    k_q k_p \Ket{k} =
    n_q n_p \Ket{k}
    ,
\end{equation}
and,
since in the spin-orbital basis the ON operators are projection operators,
they are idempotent, 
\begin{equation}
    n_p^2 =
    n_p n_p =
    a_p^{\dagger} a_p a_p^{\dagger} a_p =
    a_p^{\dagger} \left( 1 - a_p^{\dagger} a_p \right) a_p =
    a_p^{\dagger} a_p =
    n_p
    .
\end{equation}
Therefore, its eigenvalues can only be 0 or 1.

The particle-number operator, or simply the number operator, is the Hermitian
operator resulting from adding together all ON operators in the Fock space 
\begin{equation}
    \hat{N} =
    \sum_{p=1}^{m} a_p^{\dagger} a_p
    ,
\end{equation}
which returns the number of electrons in an ON vector 
\begin{equation}
    \hat{N} \Ket{k} =
    \sum_{p=1}^{m} k_p \Ket{k} =
    n \Ket{k}
    .
\end{equation}

% subsubsection Second quantization (end)
