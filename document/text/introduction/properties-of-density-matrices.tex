\subsection{Properties in terms of the density matrices} % (fold)
\label{sub:Properties in terms of the density matrices}

% subsection Properties in terms of the density matrices (end)
% As demonstrated by Gilbert, who extended the Hohenberg-Kohn theorems to nonlocal
% potentials\mycite{gilbert1975hohenbergkohn, donnelly1978elementary}, there
% exists a 1-RDM functional\mycite{gilbert1975hohenbergkohn, hohenberg1964inhomogeneousa}
% such that
Gilbert theorem\mycite{gilbert1975hohenbergkohn} grounded that the expectation
value of any observable of a system in its ground state is a unique functional
of the ground-state 1-RDM, establishing the 1-RDM as the fundamental quantity
in RDMFT in place of the electronic density on which density functional theory
(DFT) is based.
It proved that there is a 1-RDM functional\mycite{gilbert1975hohenbergkohn, hohenberg1964inhomogeneousa}
such that the Hohenberg-Kohn theorems extend to 
nonlocal potentials\mycite{gilbert1975hohenbergkohn, donnelly1978elementary}
\begin{equation} \label{ec:HK-functional}
    \func{E_{\nu}^{\text{HK}}}{\fordm} = 
    \mytrace{\hat{h} \hat{\fordm}} + 
    \Braket{\func{\Psi}{\fordm} | \hat{V}_{\text{ee}} | \func{\Psi}{\fordm}}
    ,
\end{equation}
% where $\func{\Psi}{\fordm}$ denotes a ground state wavefunction pertinent to a
% $\nu$-representable $\fordm$.
% $\hat{h}$ stands for a one-electron Hamiltonian comprising the kinetic
% energy, $\hat{T}$, and external potential, $\hat{V}_{\text{ext}}$ 
where a ground state wavefunction relevant to a $\nu$-representable $\fordm$ 
is denoted by the symbol $\func{\Psi}{\fordm}$.
The one-electron Hamiltonian, $\hat{h}$, is composed of the external
potential, $\hat{V}_{\text{text}}$, and the kinetic energy, $\hat{T}$
\begin{equation}
    \hat{h} = \hat{T} + \hat{V}_{\text{ext}}
        ,
\end{equation}
and $\hat{V}_{\text{ee}}$ is an electron interaction operator 
\begin{equation}
    \hat{V}_{\text{ee}} = 
    \sum_{i>j}^{n} \frac{1}{r_{ij}}
    .
\end{equation}

This functional follows the variational principle 
\begin{equation} \label{ec:var-princ-nu-rep}
    \func{E_{\nu}}{\fordm} \ge E_0 \quad \forall_{\fordm \in \text{$\nu$-rep}}
    ,
\end{equation}
where $\nu$-rep denotes a set of pure-state $\nu$-representable 1-RDMs.

% The domain of a density matrix functional to all pure-state $n$-representable
% 1-RDMs was extended by Levy by defining the electron repulsion functional\mycite{doi:10.1073/pnas.76.12.6062, levy1987correlation} 
Levy defined the electron repulsion functional\mycite{doi:10.1073/pnas.76.12.6062, levy1987correlation}, extending the domain of a density matrix 
functional to all pure-state $n$-representable 1-RDMs
\begin{equation} \label{ec:levy-functional}
    \func{E_{\text{ee}}^{\text{L}}}{\fordm} = 
    \min_{\Psi \to \fordm} \Braket{\Psi | \hat{V}_{\text{ee}} | \Psi}
    ,
\end{equation}
and further extended to ensemble $n$-representable 1-RDMs (belonging to a set
``$n$-rep'') by Valone\mycite{valone2008consequences, valone2008one}, so the
exact functional reads 
\begin{equation} \label{ec:ee-functional}
    \func{E_{\text{ee}}}{\fordm} =
    \min_{\sordm^{\left( n \right)}\to \fordm} \mytrace{\hat{ \mat{H}} \hat{\sordm}^{\left( n \right)}}
    ,
\end{equation}
where the minimization is carried out with respect to $n$-electron density
matrices $\sordm^{\left( n \right)}$ that yield $\fordm$.

% For a given external potential, $\hat{V}_{\text{ext}}$, minima of the Hohenberg-Kohn
% functional given in \cref{ec:HK-functional}, the Levy functional given in
% \cref{ec:levy-functional} and that in \cref{ec:ee-functional} defined, respectively,
% for $\nu$-rep, pure-state $n$-representable, and ensemble $n$-representable
% ($n$-rep) 1-RDMs, coincide\mycite{valone2008consequences, nguyen-dang1985variation}.
% Therefore, taking into account a variational principle given in \cref{ec:var-princ-nu-rep},
% one concludes that a functional defined for $n$-rep 1-RDMs yields a ground state
% energy at minimum
The minima of the Levy functional given in \cref{ec:levy-functional}, the 
Hohenberg-Kohn functional given in \cref{ec:HK-functional}, and that defined in
\cref{ec:ee-functional}, respectively for $\nu$-rep, pure-state $n$-representable,
and ensemble $n$-representable ($n$-rep) 1-RDMs coincide
for a given external potential, $\hat{V}_{\text{ext}}$\mycite{valone2008consequences, nguyen-dang1985variation}.
Thus, considering the variational principle given in \cref{ec:var-princ-nu-rep}, 
it can be concluded that a ground state energy at minimum is produced by a 
functional defined for $n$-rep 1-RDMs
\begin{equation} \label{ec:gs-energy}
    E_0 = \min_{\fordm \in \text{$n$-rep}}
    \left\{ \mytrace{\hat{h} \hat{\fordm}} + \func{E_{\text{ee}}}{\fordm} \right\}
    .
\end{equation}
The ground state energy is a functional of the 1-RDM and, therefore, it can be
obtained directly from the 1-RDM for a given electron repulsion functional,
$E_{\text{ee}}\left[ \fordm \right]$
\begin{equation}
    E \left[ \fordm \right] = 
    E_{\text{oe}} \left[ \fordm \right] +
    E_{\text{ee}} \left[ \fordm \right]
    .
\end{equation}
Indeed, \cref{ec:self-adjoint-condition,ec:normalization-condition,ec:occ-number-condition,ec:gs-energy}
are the foundation for RDMFT.

In the second quantization formalism, the electronic molecular Hamiltonian in 
the spin-orbital basis is given by
\begin{equation}
    \hat{H} =
    \sum_{pq} h_{pq} a_p^{\dagger} a_q
    +
    \frac{1}{2} \sum_{pqrs} \PBraket{pr | qs} a_p^{\dagger} a_q^{\dagger} a_s a_r
    .
\end{equation}

Using restricted spin-orbitals, it is convenient to define the operators 
\begin{equation}
    E_{pq} =
    \sum_{\sigma} a_{p\sigma}^{\dagger} a_{q\sigma}
    ,
\end{equation}
such that 
\begin{equation}
    \left[ E_{pq}, E_{rs} \right] =
    \delta_{qr} E_{ps} - \delta_{ps} E_{qr}
    .
\end{equation}

Then, the Hamiltonian can be written as 
\begin{equation}
    \hat{H} =
    \sum_{pq} h_{pq} E_{pq}
    + \frac{1}{2}
    \sum_{pqrs} \PBraket{pq | rs}\left( E_{pq} E_{rs} - \delta_{qr}E_{ps} \right)
    .
\end{equation}

% The expectation value of $\hat{H}$ with respect to a normalized reference state
% $\Ket{0}$ written as a linear combination of ON vectors 
Given a normalized reference state
$\Ket{0}$ written as a linear combination of ON vectors 
\begin{equation}
    \Ket{0} = \sum_{k} c_k \Ket{k}
    , \quad
    \Braket{0 | 0} = 1
    ,
\end{equation}
the expectation value of $\hat{H}$ with respect to $\Ket{0}$ reads
\begin{equation} \label{eq:GS-energy-1}
    E_0 = \Braket{0 | \hat{H} | 0} =
    \sum_{pq} h_{pq} D_{pq} + \frac{1}{2} \sum_{pqrs}
    \PBraket{pq | rs} \left( d_{pqrs} - \delta_{qr} D_{ps} \right)
    ,
\end{equation}
where 
\begin{equation}
    d_{pqrs} = \Braket{0 | E_{pq} E_{rs} | 0}
    ,
\end{equation}
and
\begin{equation} \label{eq:oedm-spin-representation}
    D_{pq} = \Braket{0 | E_{pq} | 0} \equiv \oedm_{pq}
    ,
\end{equation}
are the elements (densities) of an $m \times m$ Hermitian matrix, the
\textit{one-electron spin-orbital density matrix}, $ \mat{\oedm}$, since 
\begin{equation}
    D_{pq}^{*} =
    \Braket{0 | a_p^{\dagger} a_q | 0} =
    \Braket{0 | a_q^{\dagger} a_p | 0} =
    D_{qp}
    ,
\end{equation}
which is also symmetric for real wave functions.

% The one-electron density matrix is positive semidefinite since its elements are 
% either trivially equal to zero or inner products of states in the subspace
% $F\left( m, n-1 \right)$.
% The diagonal elements of $ \mat{\oedm}$ are the expectation values of the
% occupation-number operators, $n_p$, in $F\left( m,n \right)$, and are referred to as
% the occupation numbers, $\omega_p$, of the electronic state 
Since the elements of the one-electron density matrix are inner products of 
states in the subspace $F\left( m, n-1 \right)$ or trivially equal to zero, 
the one-electron density matrix is positive semidefinite.
The occupation numbers, $\omega_p$, of the electronic state are given by the
diagonal elements of $ \mat{\oedm}$, i.e. the expectation values of the 
occupation-number operators, $n_p$, in $F\left( m,n \right)$
\begin{equation}
    \omega_p = \oedm_{pp} = \Braket{0 | n_p | 0}
    .
\end{equation}

The diagonal elements of $ \mat{\oedm}$ reduce to the usual occupation numbers,
$k_p$, whenever the reference state is an eigenfunction of the ON operators, i.e.
when the reference state is an ON vector 
\begin{equation}
    \Braket{k | n_p | k} = k_p
    .
\end{equation}

Considering that the ON operators are projectors, $\omega_p$ can be written 
over the projected electronic state 
\begin{equation}
    n_p \Ket{0} = \sum_{k} k_p c_k \Ket{k}
    ,
\end{equation}
as 
the squared norm of the part of the reference state where the spin orbital
$\varphi_p$ is occupied in each ON vector
\begin{equation}
    \omega_p =
    \Braket{0 | n_p n_p | 0} =
    \sum_{k} k_p \left| c_k \right|^2
    .
\end{equation}

Recalling the $n$-representability conditions 
(\cref{ec:occ-number-condition,ec:normalization-condition}),
the occupation numbers are real numbers between zero and one 
and its sum, the trace of the density matrix, is equal to the total number
of electrons in the system 
\begin{equation}
    \trace \mat{\oedm} =
    \sum_{p} \omega_p =
    \sum_{p} \Braket{0 | n_p | 0} =
    n
    .
\end{equation}

Since $ \mat{\oedm}$ is Hermitian, it can be diagonalized by the spectral
decomposition for a unitary matrix, $ \mat{U}$, as
\begin{equation} \label{eq:diagonalization-oedm}
    \mat{\oedm} =
    \mat{U} \lambda \mat{U}^{\dagger}
    .
\end{equation}

The eigenvalues, $\lambda_p$, are real numbers 
\begin{equation}
    0 \le \lambda_p \le 1
    ,
\end{equation}
known as the natural-orbital occupation numbers, which also fulfill 
\begin{equation}
    \sum_{p} \lambda_p = n
    .
\end{equation}

A new set of spin orbitals, the natural spin orbitals, are obtained from the
eigenvectors, the columns of $ \mat{U}$, resulted from 
\cref{eq:diagonalization-oedm}.

Following the idea given in \cref{ec:ee-functional} and 
with the definition of the one-electron density matrix given in 
\cref{eq:oedm-spin-representation}, \cref{eq:GS-energy-1} can be further
simplified introducing the \textit{two-electron spin-orbital density matrix}
\begin{equation} \label{eq:tedm-spin-representation}
    \tedm_{pqrs} = d_{pqrs} - \delta_{qr} D_{ps}
    ,
\end{equation}
as
\begin{equation} \label{eq:GS-energy-2}
    E_0 =
    \sum_{pq} h_{pq} \oedm_{pq} + \frac{1}{2} \sum_{pqrs}
    \PBraket{pq | rs} \tedm_{pqrs}
    .
\end{equation}

Then, due to the two-particle nature of the electron interaction functional
given in \cref{ec:ee-functional}, the contraction of the 2-RDM matrix components,
$\tedm_{pqrs}$, with two-electron integrals $\PBraket{pq | rs}$, the
electronic repulsion energy $E_{\text{ee}}$ may be approximated as
\begin{equation} \label{eq:Eee_from_Gamma}
    E_{\text{ee}} \left[ \fordm \right]
    =
    \frac{1}{2} \sum_{pqrs}
    \sordm_{pqrs}\left[ \left\{ n_t \right\} \right] 
    \Braket{pq | rs}
    ,
\end{equation}
where the 2-RDM is formally defined as 
\begin{equation}
    \tedm_{pqrs} = \Braket{0 | a_r^{\dagger} a_s^{\dagger} a_q a_p | 0}
    .
\end{equation}
Therefore, the 2-RDM is formally a functional of the 1-RDM. %TODO: incluir alguna cita

The elements of $ \mat{\tedm}$ are not all independent because of the
anticommutation relations between the creation and annihilation operators 
\begin{equation}
    \tedm_{pqrs} = - \tedm_{rqps} = - \tedm_{ps rq} = \tedm_{rspq}
    .
\end{equation}

Also, in accordance with the Pauli principle 
\begin{equation}
    \tedm_{pqps} = \tedm_{pqrq} = \tedm_{pqpq} = 0
    .
\end{equation}

The two-electron density matrix can be rewritten as a
$m\left( m-1 \right) /2 \times m\left( m-1 \right) /2$ 
matrix, $ \mat{T}$ 
\begin{equation}
    T_{pq,rs} = \Braket{0 | a_p^{\dagger} a_q^{\dagger} a_s a_r | 0}
    , \quad
    p>q,\ r>s
    ,
\end{equation}
with composite indices $pq$, composing a subset of $ \mat{\tedm}$ by a
reordering of the middle indices 
\begin{equation}
    T_{pq,rs} = \tedm_{prqs}
    , \quad
    p>q,\ r>s
    .
\end{equation}

The matrix $ \mat{T}$ is also Hermitian and, therefore, symmetric for real
wave functions.
Furthermore, it is positive semidefinite since its elements are either zero
or inner products of states in $F\left( m, n-2 \right)$.

The diagonal elements of $ \mat{T}$, considering that $p > q$ and introducing
the ON operators, are given by
\begin{equation}
    \omega_{pq} = T_{pq,pq} =
    \Braket{0 | a_p^{\dagger} a_q^{\dagger} a_q a_p | 0} =
    \Braket{0 | n_p n_q | 0},
\end{equation}
which can be interpreted as simultaneous occupations of pairs of spin orbitals
from the part of the wave function where the spin
orbitals $\varphi_p$ and $\varphi_q$ are simultaneously occupied.
It can be simply denoted as pair occupations, and fulfill the following conditions:
\begin{enumerate}
        % Simultaneous occupation of a given spin-orbital pair cannot exceed those of
        % the individual spin orbitals 
    \item The simultaneous occupation of a given spin-orbital cannot be greater than that
        of the individual spin orbitals
        \begin{equation}
            0 \le \omega_{pq} \le \min\left[ \omega_p, \omega_q \right] \le 1
            .
        \end{equation}

    \item The sum of all pair occupations $\omega_{pq}$ is equal to the number of
        electron pairs in the system 
        \begin{equation}
            \trace \left[ \mat{T} \right] =
            \sum_{p>q} \Braket{0 | n_p n_q | 0} =
            \frac{1}{2} \sum_{pq} \Braket{0 | n_p n_q | 0}
            - \frac{1}{2} \sum_{p} \Braket{0 | n_p | 0} =
            \frac{1}{2} n \left( n - 1 \right)
            .
        \end{equation}
\end{enumerate}

% For a state containing a single ON vector, $\Ket{k}$, the $ \mat{T}$ matrix
% has a simple diagonal structure and may be constructed directly from the
% one-electron density matrix as
The $ \mat{T}$ matrix for a state with a single ON vector, $\Ket{k}$, has a 
straightforward diagonal form and may be constructed directly from the one-
electron density matrix as
\begin{equation} \label{eq:T_from_D1s}
    T_{pq,rs}^{\Ket{k}} = D_{pq}^{\Ket{k}} D_{qs}^{\Ket{k}}
    .
\end{equation}
% and, likewise, the expectation value of any one- and two-electron operator
% may be obtained directly from the one-electron density matrix.
Similarly, the one-electron density matrix can be used to obtain the 
expectation value of any one or two-electron operator.

Also, the 1-RDM can be reconstructed from the 2-RDM by taking the trace of the
2-RDM over one of the indices, i.e. summing over all the possible remaining
electrons after selecting one 
\begin{equation}
    \oedm_{pq} =
    \frac{1}{N-1} \sum_{r} \tedm_{prqr}
    .
\end{equation}

% This observation is consistent with the picture of an uncorrelated description
% of the electronic system where the simultaneoys occupations of pairs of spin
% orbitals are just the products of the individual occupations.
% For a general electronic state containing more than one ON vector , $ \mat{T}$ 
% is in general not diagonal and cannot be generated directly from the one-electron
% density elements. In practice, it can be diagonalized with an unitary matrix
% as shown for the one-electron density matrix.
The picture of an uncorrelated description of the electronic system, in which 
the simultaneous occupations of pairs of spin orbitals are merely the products 
of the individual occupations, is compatible with this result.
In the case of a general electronic state containing several ON vectors, $\mat{T}$
is generally non-diagonal and cannot be produced directly from the elements of 
the one electron density matrix. As demonstrated for the one-electron density 
matrix in \cref{eq:diagonalization-oedm}, $ \mat{\tedm}$ can be diagonalized
in practice using the spectral decomposition.

Therefore, it can be stated that the one-electron density matrix probes the
individual occupancies of the spin orbitals and describes how the $n$ electrons
are distributed among the $m$ spin orbitals, whereas the two-electron density
matrix probes the simultaneous occupations of the spin orbitals and describes
how the $n\left( n-1 \right) /2$ electron pairs are distributed among the 
$m\left( m-1 \right) /2$ spin-orbital pairs\mycite{helgaker2000second}.

