% ---------
\subsection{Minkowski distance}
% ---------
Also, a second metric is computed to directly compare the approximated 2-RDMs.
Taking one element of one 2-RDM as a point in the high-dimensional 4D space,
the Minkowski distance can be computed between a point in the 4D space given by the
exact 2-RDM and a point by the approximated 2-RDM.

The Minkowski distance (metric) of order $p$ for an approximated 2-RDM,
$ \mat{\tilde{\sordm}}$, with respect to the exact 2-RDM, $ \mat{\sordm}$,
is given by 
\begin{equation} \label{eq:minkowski-distance}
    d \left( \mat{\sordm} , \mat{\tilde{\sordm}} \right) =
    \left(
        \sum_{pqrs} \left| \sordm_{pqrs} - \tilde{\sordm}_{pqrs} \right|^p
    \right)^{\frac{1}{p}}
    ,
\end{equation}
which reduces to the Manhattan distance for $p=1$, to the Euclidean distance
for $p=2$ and, in the limit, to the Chebyshev distance for $p \to \infty$.
In practice, several orders should be tried for each case, looking for the
order that results in largest differences between the distances that are
being compared.

Similarly to the energy, the Minkowski distance can be splitted into
inactive, active and cross contributions by adjusting the indexes 
\begin{equation}
    d \left( \mat{\sordm} , \mat{\tilde{\sordm}} \right) =
    d_{\text{ina}} + d_{\text{i-a}} + d_{\text{act}} =
    \left(
        \sum_{ijkl}^{n_{\text{io}}} \left| \Delta \sordm_{ijkl} \right|^p
    \right)^{\frac{1}{p}}
    +
    \left(
        \sum_{ij}^{n_{\text{io}}} \sum_{tu}^{n_{\text{ao}}} \left| \Delta \sordm_{ijtu} \right|^p
    \right)^{\frac{1}{p}}
    +
    \left(
        \sum_{tuvw}^{n_{\text{ao}}} \left| \Delta \sordm_{tuvw} \right|^p
    \right)^{\frac{1}{p}}
    ,
\end{equation}
in order to have a better understaing of each block.

Therefore, the Minkowski distance is a direct metric to test any approximated
2-RDM, resulting in a similar 2-RDM if $d \to 0$, and a non-similar 2-RDM
for larger $d$'s.
In this sense, this metric has no value if only one approximated 2-RDM is tested,
as this distance has to be compared with other distances in order to establish
the quality of the approximation.
Also, it is not a direct metric of the performance of a 2-RDM.
