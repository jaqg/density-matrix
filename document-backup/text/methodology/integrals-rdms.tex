% ---------
\subsection{One- and two-electron integrals, exact 1-RDM and 2-RDM}
% ---------

The one-electron integrals matrix, $h_{pq}$, is extracted from the 
\inline{AOONEINT} file with the \inline{ONEHAMIL} label from the Dalton
software\mycite{aidas2014dalton}.
Leaving the technical parts behind, the lower triangular is read in the atomic
orbital (AO) basis packed by symmetry, and later umpacked to
obtain the full matrix and symmetrized as $h_{ji} = h_{ij}$.
It is converted from atomic orbital basis, $\tilde{h}_{ab}$, to the molecular
orbital basis, $h_{ij}$, as 
\begin{equation}
    \mat{h} = \mat{C}^{\dagger} \mat{\tilde{h}} \mat{C}
    \implies
    h_{ij} = C_{ai}^{\dagger} \tilde{h}_{ab} C_{bj},
\end{equation}
where $ \mat{C}$ is the MO coefficients matrix.

The two-electron integrals matrix, $ \PBraket{pq | rs}$, is extracted from the
\inline{MOTWOINT} with the \inline{MOLTWOEL} label already in the MO basis and
written in the Mulliken (chemistry) notation.
Similarly, it is unpacked to full dimension and symmetrized considering the
8-fold symmetry
\begin{equation} \label{eq:spac-int-8fold}
    \PBraket{ij | kl} = 
    \PBraket{ij | lk} = 
    \PBraket{ji | kl} = 
    \PBraket{ji | lk} = 
    \PBraket{kl | ij} = 
    \PBraket{kl | ji} = 
    \PBraket{lk | ij} = 
    \PBraket{lk | ji}
    .
\end{equation}

Then, along other data, the density matrices $\mat{\oedm}$ and $\mat{\tedm}$ are
extracted from the interface \inline{SIRIFC} file similarly to the integrals
matrices.
Once symmytrized, since the density matrices are already in the (molecular)
spacial orbitals, they are refered to as $\mat{\fordm}$ and $\mat{\sordm}$.

For the CASSCF calculations, the complete 1-RDM is constructed by block matrices
corresponding to each -inactive, active, virtual- subspace
\begin{equation} \label{eq:complete-cas-1rdm-1}
    \mat{\fordm} =
    \begin{pmatrix}
        \mat{\fordm}^{\text{i-i}} & \mat{\fordm}^{\text{i-a}} & \mat{\fordm}^{\text{i-v}} \\
        \left( \mat{\fordm}^{\text{i-a}} \right)^{\dagger} & \mat{\fordm}^{\text{a-a}} & \mat{\fordm}^{\text{a-v}} \\
        \left( \mat{\fordm}^{\text{i-v}} \right)^{\dagger} & \left( \mat{\fordm}^{\text{a-v}} \right)^{\dagger} & \left( \mat{\fordm}^{\text{v-v}} \right)^{\dagger} \\
    \end{pmatrix}
    ,
\end{equation}
where i,a,v superindexes correspond to inactive, active and virtual orbitals,
respectively.
Also, $\mat{\fordm}^{\dagger} = \mat{\fordm}$ as all its elements
are real.

Since the inactive orbitals are totally occupied, the inactive-inactive block
matrix is a diagonal matrix with diagonal elements equal to the total occupation,
with a value of two considering double occupation
\begin{equation}
         \fordm_{ij}^{\text{i-i}} = 2\delta_{ij}
         \implies
         \mat{\fordm}^{\text{i-i}} = 2\mat{I}
         ,
\end{equation}
where $ \mat{I}$ is the identity matrix.
Then, $ \mat{\fordm}^{\text{i-i}}$ is already symmetric.

The active-active block matrix is the one obtained from Dalton which is,
similarly to $\mat{h}$, symmytrized as 
$\oedm_{ji} = \oedm_{ij} \equiv \fordm_{ij}^{\text{a-a}}$.

Since the virtual orbitals are not occupied, both the diagonal virtual-virtual
block matrix and the crossed virtual-inactive and virtual-active block
matrices are zero 
\begin{equation}
    \mat{\fordm}^{\text{v-v}} =
    \mat{\fordm}^{\text{i-v}} =
    \mat{\fordm}^{\text{i-v}} = 0
    ,
\end{equation}
and, also, the inactive-active block matrix is zero assuming that there is no
direct correlation betweem inactive and active orbitals 
\begin{equation}
    \mat{\fordm}^{\text{i-a}} = 0.
\end{equation}

Therefore, the complete 1-RDM given in \cref{eq:complete-cas-1rdm-1} reads
\begin{equation} \label{eq:complete-cas-1rdm-2}
    \mat{\fordm} =
    \begin{pmatrix}
        2\mat{I} & 0 & 0 \\
        0 & \mat{\fordm}^{\text{a-a}} & 0 \\
        0 & 0 & 0 \\
    \end{pmatrix}
    .
\end{equation}

In principle, $ \mat{\fordm}$ is already diagonal, as $ \mat{\fordm}^{\text{a-a}}$
is already diagonalized in Dalton.
To generalize, a Jacobi diagonalization routine is implemented to diagonalize
$ \mat{\fordm}$ if it is not diagonal. 
But, an unitary transformation of the integral matrices with the eigenvectors
in order to work with the natural orbitals basis has not been implemented yet.
Then, in addition to the symmetry restrictions, the normalization condition
(\cref{ec:normalization-condition}) is asured to be fulfilled.

Analogously, the complete 2-RDM must be constructed and symmetrized
to be consistent with the 8-fold symmetry of the
two-electron integrals (see \cref{eq:spac-int-8fold})
\begin{equation} \label{eq:sym-2rdm}
    \sordm_{\alpha \beta \gamma \delta } = 
    \sordm_{\alpha \beta \delta \gamma } = 
    \sordm_{\beta \alpha \gamma \delta } = 
    \sordm_{\beta \alpha \delta \gamma } = 
    \sordm_{\gamma \delta \alpha \beta } = 
    \sordm_{\gamma \delta \beta \alpha } = 
    \sordm_{\delta \gamma \alpha \beta } = 
    \sordm_{\delta \gamma \beta \alpha }
    .
\end{equation}

It must fulfill the following symmetry conditions
\begin{align}
    \label{eq:sym-cond-1-2rdm}
    \sordm_{\beta\alpha\gamma\delta} = \sordm_{\alpha\beta\gamma\delta},\ \ & \alpha \leftrightarrow \beta ,\\
    \label{eq:sym-cond-2-2rdm}
    \sordm_{\beta\alpha\delta\gamma} = \sordm_{\alpha\beta\gamma\delta},\ \ & \gamma \leftrightarrow \delta ,\\         
    \label{eq:sym-cond-3-2rdm}
    \sordm_{\gamma\delta\alpha\beta} = \sordm_{\alpha\beta\gamma\delta},\ \ & \alpha\beta \leftrightarrow \gamma \delta , 
\end{align}
for any index $\alpha, \beta, \gamma, \delta$.
To simplify the formulation, the definitions for the non-zero elements are
directly given\mycite{maradzike2017analytic}, where inactive orbitals are
inndexed with $i,j,k,l$ and active orbitals with $t,u,v,w$.

The elements of the 2-RDM involving only inactive orbitals are given by 
\begin{equation}
    \tedm_{ijkl} = 
    2 \delta_{ij} \delta_{kl} - \delta_{il} \delta_{jk}
    ,
\end{equation}
which are then symmetrized as 
\begin{equation}
    \sordm_{ijkl} =
    \frac{1}{2} \left( \tedm_{ijkl} + \tedm_{jikl} \right)
    ,
\end{equation}
fulfilling \cref{eq:sym-cond-1-2rdm,eq:sym-cond-2-2rdm,eq:sym-cond-3-2rdm}, 
$\sordm_{jikl} = \sordm_{jilk} = \sordm_{lkij} = \sordm_{ijkl}$.

The elements of the 2-RDM involving only active orbitals are obtained from Dalton,
where a packed matrix $\tilde{D}_{tu,vw}^{(2)}$ with composite indexes $tu,vw$ is
constructed as half of the values obtained from Dalton\footnote{A computational
    detail is that Dalton writes $\mat{\tilde{D}^{(2)}}$ multiplied by two to
    reduce the dimensions of the loops taking advantage of the symmetry of the
    integrals (see \cref{eq:spac-int-8fold}) and of $\mat{\tilde{D}^{(2)}}$, as 
    $\tilde{D}_{tu,vw}^{(2)} = \tilde{D}_{vw,ut}^{(2)}$,
    to sum only over the lower triangular part of $ \mat{\tilde{D}^{(2)}}$.
} and then
unpacked to full dimension $\tedm_{tuvw}$.
It should already be symmetrized by Dalton, $\sordm_{tuvw} = \sordm_{uvwt}$, and
must fulfill
$\sordm_{utvw} = \sordm_{tuwv} = \sordm_{vwtu} = \sordm_{tuvw}$.
% $\sordm_{utvw} = \sordm_{tuvw}$ by $t$-$u$ swapping;
% $\sordm_{tuwv} = \sordm_{tuvw}$ by $v$-$w$ swapping;
% and $\sordm_{vwtu} = \sordm_{tuvw}$ by $tu$-$vw$ swapping.

The elements with two active-two inactive orbitals are proportional to the
1-RDM for the active orbitals and correspond to Coulomb and exchange terms.
The Coulomb terms are given by
\begin{equation} \label{eq:coulomb-term-2rdm}
    \tedm_{ijtu} = \tedm_{tuij} = 2 \fordm_{tu} \delta_{ij},
\end{equation}
which are symmetrized as 
\begin{equation}
    \sordm_{tuij} = \sordm_{ijtu} =
    \frac{1}{2} \left( \tedm_{tuij} + \tedm_{ijtu} \right)
    ,
\end{equation}
and fulfill
$\sordm_{utij} = \sordm_{tuji} = \sordm_{ijtu} = \sordm_{tuij}$.
% $\sordm_{utij} = \sordm_{tuij}$ by $t$-$u$ swapping;
% $\sordm_{tuji} = \sordm_{tuij}$ by $i$-$j$ swapping;
% $\sordm_{ijtu} = \sordm_{tuij}$ by $tu$-$ij$ swapping.

And the exchange terms by
\begin{equation} \label{eq:exchange-term-2rdm}
    \tedm_{iutj} = \tedm_{tjiu} = - \fordm_{tu} \delta_{ij}
    ,
\end{equation}
symmetrized as
\begin{equation}
    \sordm_{tjiu} = \sordm_{jtiu} =
    \frac{1}{2} \left( \tedm_{tjiu} + \tedm_{jtiu} \right)
    ,
    % \quad 
    % \sordm_{tjui} = 
    % \frac{1}{2} \left( \tedm_{tjiu} + \tedm_{tjui} \right)
    % ,
\end{equation}
and fulfill
$\sordm_{utij} = \sordm_{tuji} = \sordm_{ijtu} = \sordm_{tuij}$.

The 2 and $-1$ factors for the Coulumb term, $\sordm_{ijtu}$ (\cref{eq:coulomb-term-2rdm}),
and exchange, $\sordm_{iutj}$ (\cref{eq:exchange-term-2rdm}), differ in this
case from the 1, $-\frac{1}{2}$ factors given by Maradzike\mycite{maradzike2017analytic}.

In addition to the symmetry restrictions, the normalization conditions in
\cref{ec:normalization-condition,ec:normalization-condition-2RDM} are asured
to be fulfilled.
