% ---------
\subsection{Energy calculations}
% ---------
The ground state energy, with \cref{eq:GS-energy-2}, is computed as
\begin{equation}
    E_0 = 
    \sum_{pq} h_{pq} \fordm_{pq}
    +
    \sum_{pqrs}
    \sordm_{pqrs} \PBraket{pq | rs}
    + V_{\text{nuc}}
    ,
\end{equation}
where $V_{\text{nuc}}$ is the nuclear repulsion potential.

In order to have a better understanding, the energy is splitted into the
inactive, active-inactive (cross) and active contributions
\begin{equation} \label{eq:E0-splitted}
    E_0 =
    E_{\text{ina}} + E_{\text{i-a}} + E_{\text{act}} + V_{\text{nuc}}
    .
\end{equation}

Considering the one- and two-electron contributions separately, as 
\begin{equation}
    E_0 =
    E_{\text{oe}} + E_{\text{ee}} + V_{\text{nuc}}
    ,
\end{equation}
the one-electron and two-electron functionals are splitted independently as
\begin{equation} \label{eq:Eoe-splitted}
    E_{\text{oe}} = 
    E_{\text{oe}}^{\text{ina}} + E_{\text{oe}}^{\text{i-a}} + E_{\text{oe}}^{\text{act}}
    ,
\end{equation}
with
\begin{align}
    E_{\text{oe}}^{\text{ina}}
    &=
    \sum_{ij}^{n_{\text{io}}} h_{ij} \fordm_{ij}
    ,
    \\
    E_{\text{oe}}^{\text{i-a}}
    &=
    \sum_{i}^{n_{\text{io}}} \sum_{t}^{n_{\text{ao}}} h_{it} \fordm_{it}
    ,
    \\
    E_{\text{oe}}^{\text{act}}
    &=
    \sum_{tu}^{n_{\text{ao}}} h_{tu} \fordm_{tu}
    ,
\end{align}
where $n_{\text{io}}$ and $n_{\text{ao}}$ are the number of inactive and
active orbitals, respectively.
In this case, although, $E_{\text{oe}}^{\text{i-a}} = 0$.

And, similarly for the two-electron contributions
\begin{equation} \label{eq:Eee-splitted}
    E_{\text{ee}} = 
    E_{\text{ee}}^{\text{ina}} + E_{\text{ee}}^{\text{i-a}} + E_{\text{ee}}^{\text{act}}
\end{equation}
with
\begin{align}
    E_{\text{ee}}^{\text{ina}}
    &=
    \sum_{ijkl}^{n_{\text{io}}} 
    % \sordm_{ijkl} \PBraket{ij | kl}
    P_{ijkl}
    ,
    \\
    E_{\text{ee}}^{\text{i-a}}
    &=
    \sum_{ij}^{n_{\text{io}}} \sum_{tu}^{n_{\text{ao}}}
    \left( 
        % \sordm_{ijtu} \PBraket{ij | tu} +
        P_{ijtu} +
        % \sordm_{tuij} \PBraket{tu | ij} +
        P_{tuij} +
        % \sordm_{tjiu} \PBraket{tj | iu} +
        P_{tjiu} +
        % \sordm_{jtiu} \PBraket{jt | iu} +
        P_{jtiu} +
        % \sordm_{tjui} \PBraket{tj | ui} +
        P_{tjui} +
        % \sordm_{jtui} \PBraket{jt | ui}
        P_{jtui}
    \right)
    ,
    \\
    E_{\text{ee}}^{\text{act}}
    &=
    \sum_{tuvw}^{n_{\text{ao}}} 
    % \sordm_{tuvw} \PBraket{tu | vw}
    P_{tuvw}
    ,
\end{align}
where $P_{ijkl} = \sordm_{ijkl} \PBraket{ij | kl}$ for the symmetrized 
$ \mat{\sordm}$.

Lastly, the following relations between \cref{eq:Eoe-splitted,eq:Eee-splitted}
with \cref{eq:E0-splitted} hold 
\begin{align}
    E_{\text{ina}} &=
    E_{\text{oe}}^{\text{i-a}} + E_{\text{ee}}^{\text{i-a}}
    , \\
    E_{\text{i-a}} &= 
    E_{\text{oe}}^{\text{act}} + E_{\text{ee}}^{\text{i-a}}
    , \\
    E_{\text{act}} &= 
    E_{\text{i-a}} + E_{\text{ee}}^{\text{act}} =
    E_{\text{oe}}^{\text{act}} + E_{\text{ee}}^{\text{i-a}} + E_{\text{ee}}^{\text{act}}
    .
\end{align}
