\subsection{Reduced Density Matrix Functional Theory} % (fold)

% subsubsection 1-RDM (end)
Many methods of electronic structure theory are based on variational
optimization of an energy functional.
An idempotent first-order reduced density matrix (1-RDM) is variationally
optimized in the Hartree-Fock (HF) method\mycite{kollmar2004structure}.

The 1-RDM is determined by the natural orbitals and their occupation
numbers\mycite{lowdin1955quantum}, leading to denote the corresponding energy
functional as density matrix functional (DMF) or natural orbital functional (NOF).
This is explained in detail in the following sections.

% \subsubsection{1-RDM} % (fold)
% \label{sec:1-RDM}

% subsubsection 1-RDM (end)
% NOTAS SACADAS DEL PERNAL
% mirar tb seccion 1.7.3 del Hergakel
In the coordinate representation of first quantization,
the \textit{first-order reduced density matrix} (1-RDM), also called
\textit{one-particle reduced density matrix}, $\fordm$, is defined
for an $n$-electron wavefunction, $\Psi$, as
\begin{equation} \label{eq:fordm-coordinate-representation}
    \fordm \left( \mat{x}_1, \mat{x}_1\myprime \right) =
    n \idotsint
    \Psi \left( \mat{x}_1, \mat{x_2}, \ldots,  \mat{x}_n \right)
    \Psi^{*} \left( \mat{x}_1\myprime, \mat{x_2}, \ldots,  \mat{x}_n \right)
    \dd{ \mat{x}_2} \cdots \dd{ \mat{x}_n}
    ,
\end{equation}
and the \textit{second-order reduced density matrix} (2-RDM), or
\textit{two-particle reduced density matrix}, $\sordm$, as 
\begin{equation} \label{eq:sordm-coordinate-representation}
    \sordm \left( \mat{x}_1, \mat{x}_2, \mat{x}_1\myprime, \mat{x}_2\myprime \right)
    =
    \frac{n\left( n - 1 \right)}{2} \idotsint
    \Psi \left( \mat{x}_1, \mat{x}_2, \mat{x}_3, \ldots,  \mat{x}_n \right)
    \Psi^{*} \left( \mat{x}_1\myprime, \mat{x}_2\myprime, \mat{x}_3, \ldots,  \mat{x}_n \right)
    \dd{ \mat{x}_3} \cdots \dd{ \mat{x}_n}
    ,
\end{equation}
where $ \mat{x} = \left( \mat{r}, s \right)$ is a combined spatial and spin
coordinate.

%TODO: incluir info de seccion 2 del pernal

% An inmidiate advantage of using 1-RDM instead of the electron density, $\rho$,
% is that the kinetic energy is an explicit functional of $\fordm$ but not of
% $\rho$.
Employing the 1-RDM rather than the electron density, $\rho$, has the immediate
advantage of making the kinetic energy an explicit functional of $\fordm$ rather
than $\rho$.

% Then, there is no need to introduce a ficticious noninteracting system.
Then, introducing a fictitious noninteracting system is not necessary.

% Moreover, orbitals present in Reduced Density Matrix Functional Theory (RDMFT)
% are fractionally occupied, so functionals 
% of $\fordm$ seem to be better suited to account for static correlation.
Furthermore, fractionally occupied orbitals are a feature of Reduced Density 
Matrix Functional Theory (RDMFT), which suggests that functionals of $\fordm$ are 
better suited to explain static correlation.

Self-adjointness of $\fordm$, as defined in \cref{eq:fordm-coordinate-representation},
allows for its spectral representation\mycite{lowdin1955quantum}
\begin{equation}
    \fordm \left( \mat{x}, \mat{x}\myprime \right) =
    \sum_{p} n_p 
    \varphi_p \left( \mat{x} \right)
    \varphi_p^{*} \left( \mat{x}\myprime \right)
    .
\end{equation}

The eigenvalues of 1-RDM are called natural occupation numbers,
$\left\{ n_p \right\}$, and the eigenfunctions are known as natural
spin-orbitals, $\left\{ \varphi_p \right\}$.
% For convention, the indices $p,q,r,s$ are referred to natural spin-orbitals
% and $a,b,c,d$ to arbitrary one-eletron functions.
% Also, atomic units are employed throughout the document.
By convention, natural spin-orbitals are denoted by the indices $p$, $q$, $r$, and $s$, 
and arbitrary one-eletron functions by the indices $a$, $b$, $c$, and $d$. 

% Some properties of the natural spin-orbitals and occupation numbers, known as the 
% $n$-representability conditions, are:
% the ensemble n-representability conditions for the first-order reduced density
% matrix have a particularly simple form\mycite{coleman1963structure}
The ensemble $n$-representability conditions, which are some properties of the 
natural spin-orbitals and occupation numbers, have a particularly simple 
form\mycite{coleman1963structure} and read:
\begin{enumerate}
    \item Self-adjointness of $\fordm$: implies orthonormality of the natural orbitals 
        \begin{equation} \label{ec:self-adjoint-condition}
            \int
            \varphi_p^{*} \left( \mat{x} \right)
            \varphi_q \left( \mat{x} \right)
            \dd{ \mat{x}}
            = \delta_{pq}
            \quad \forall_{p,q} 
            .
        \end{equation}
        
    \item By the Löwdin's normalization convention\mycite{lowdin1955quantum},
        $\fordm$ is assumed to be normalized to a number of electrons, $n$.
        Therefore, the natural occupancies sum up to $n$  
        \begin{equation} \label{ec:normalization-condition}
            \trace \left[ \fordm \right] =
            \sum_{p} \fordm_{pp} =
            \sum_{p} n_p =
            n
            .
        \end{equation}

    \item The occupation number, $n_p$, is nonnegative and not grater than 1 
    \begin{equation} \label{ec:occ-number-condition}
        0 \leq n_p \leq 1 \quad \forall_{p}
        .
    \end{equation}
\end{enumerate}

