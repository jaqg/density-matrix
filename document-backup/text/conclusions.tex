% ------------------------------
\section{Conclusions and future perspectives} % (fold)
\label{sec:conclusions}
% ------------------------------
In this thesis, the application of Reduced Density Matrix Functional Theory 
(RDMFT) for electronic structure calculations has been extensively explored.
The primary goal was to create a working environment that enables the 
following optimization of the possible new parameterization proposed in this
work.

Several key findings emerged from this study:
\begin{enumerate}[label={}]
    \item \textit{First.} Validation of approximations:
        Stablished approximations for the electron repulsion functional in terms of 
        the two-particle reduced density matrix (2-RDM) have been validated.
        These approximations have shown promise in accurately capturing 
        electron correlation effects in small sized systems, as in this case,
        \ch{H2O}.
    \item \textit{Second.} Possible new parameterization:
        A possible new parameterization of the 2-RDM was proposed, providing 
        possible approaches and algorithms for a future optimization.
    \item \textit{Third.} Computational implementation:
        The methodologies proposed were implemented and tested at
        various theory levels, and is extensible to other systems
        simply changing the Dalton input.
        Their feasibility and effectiveness in practical applications was
        demonstrated.
\end{enumerate}

Reduced Density Matrix Functional Theory presents a promising improvement
in computational cost to wavefunction-based methods.
Therefore, the study of reduced density matrices and their applications
presents multiple avenues for future research.
One of the promising directions is the development of new approximations
and improvement of existing approximations for the electron repulsion functional,
specially for its application to larger systems.

The development of more precise approximated reduced 
density matrices and efficient methods directly repercute in accuratly describing
electronic properties, which has significant implications in fields such as
computational chemistry and solid-state physics. 
% The ability to accurately describe the energy and properties of complex 
% electronic systems can enhance the design of materials and molecules with 
% desired properties, such as more efficient catalysts or materials with 
% specific electronic characteristics

There are a vast of interdisciplinary applications of RDMFT, such as
to molecular biology with the study of quantum-level interactions in biomolecules.
Along those practical applications, RDMFT is used in theoretical physics
to provide new insights into the nature of electronic interactions in strongly
correlated systems.

The principal advantage of using reduced density matrices over classical
wavefunction methods lies in their ability to capture electronic correlation 
without an explicit representation of the 
full many-body wavefunction, which removes exponential grow with the number of 
electrons and orbitals.
The 2-RDM carries all the relevant information if one is interested in
expectation values of one- and two-particle operators, which is almost always
the case.
Therefore, the 2-RDM is a much more compact and economic storage of
information than the $N$-particle wave function.
This reduction in complexity could make such methods computationally feasible
for larger systems and provide a more efficient means of handling electron
correlation effects.
By using functionals of the reduced density matrices, RDMFT can offer a 
balance between accuracy and computational feasibility, making it a powerful 
tool for studying complex electronic systems.

One of the main challenges is that the 2-RDM has to obey complicated
$N$-representability conditions in order to correspond its formal definition
(\cref{eq:sordm-coordinate-representation}), while the wave function only has
to be antisymmetric and normalized to unity.
This results in a high computational complexity problem associated with accurate
approximations of reduced density matrices.
Additionally, one of the main downsides of current methods is their 
application primarily to small systems, for which approximated functionals
have been parametrized.
This limitation arises due to the exponential increase in computational 
resources required as the system size grows, specially for the 2-RDM as it
is constructed as a 4-dimensional tensor.
For larger systems, the computational cost becomes prohibitive because the 
number of possible electronic configurations increases dramatically.

However, with advances in quantum computing and machine learning algorithms,
there is a significant opportunity to overcome these limitations.
For example, the optimization of parametrization factors, such as $\mu$ and
$\mu^{\prime}$ in \cref{eq:AQ-approximation}, with type-algorithms as those
proposed in this work involve expensive methods and algorithms as matrix and
tensor decompositions and diagonalization.
Therefore, the integration of this variational-type methods in quantum computing,
which is a currently developing area, would overcome this problems. 
Also, the integration of machine learning techniques to this kind of 
parametrizations and optimizations could improve the cost and, more importantly,
the application to larger and more complex systems.
% The integration of machine learning techniques could enable the prediction of
% electronic properties more quickly and at a lower computational cost

Over the past few decades, there has been extensive theoretical development 
around reduced density matrices.
This includes the formulation and refinement of the contracted Schrödinger 
equations (CSE), which offer a framework for approximating the many-body 
problem by considering only the RDMs of interest.
% The CSEs have been pivotal in advancing our understanding and computational 
% approaches to electronic correlation in quantum systems.

While this document has primarily focused on the 1-RDM and 2-RDM due to their 
importance, physical significance and relatively simpler formal definitions,
there have been substantial theoretical advancements concerning higher-order
reduced density matrices, such as the 3-RDM and 4-RDM, which are of the most
practical interest.
Also, analogously to the reconstruction of the 1-RDM from the 2-RDM given in
\cref{eq:1rdm-from-2rdm}, lower-order RDMs can be reconstructed from higher-order
ones by tracing over a dimension.
These higher-order RDMs provide even more detailed information about 
electronic interactions but come with increased complexity in both definition 
and computational demand.
Exploring these higher-order RDMs can potentially lead to more accurate 
descriptions of electronic systems and new insights into many-body quantum 
phenomena.

The primary future vision for this work is for the potentially new 
approximation proposed in this document to be verified and optimized. 
By achieving this, the program is intended to be enhanced, generalized, and 
made more efficient, ultimately seeking its possible integration with existing 
software such as Dalton.

Additionally, since no time-dependence has been considered in this work, 
an important extension would be the development and application to Time-Dependent
Reduced Density Matrix Functional Theory (TDRDMFT), which is a relatively 
young extension of RDMFT and represent a significant future research direction.
TDRDMFT can provide insights into the dynamic behavior of 
electronic systems under various external perturbations, enhancing our 
understanding of time-dependent processes in complex quantum systems with
the advantage of RDMFT.

Looking forward, several future research directions have been identified:
\begin{itemize}
    \item Time-Dependent Extensions:
        Extending the current methodologies to Time-Dependent Reduced Density 
        Matrix Functional Theory (TDRDMFT) could provide valuable insights 
        into dynamic processes in electronic systems.
    \item Theoretical Advancements:
        Optimization of the proposed parametrization and testing.
    \item Integration with Existing Software:
        Integration of the program with existing software such as Dalton.
\end{itemize}

In conclusion, this thesis provides a comprehensive framework for employing 
reduced density matrice in electronic structure calculations, highlighting both
its potential and the challenges that need to be addressed.
The field of Reduced Density Matrix Functional Theory and RDM approximations is
full of opportunities and challenges.
It is essential to continue improving existing 
approximations and methodologies and exploring new techniques that can overcome current limitations.
With the convergence of different disciplines and the use of emerging 
technologies, the future of this field promises to be exciting and full of 
innovative discoveries.
The advancements made here lay the groundwork for further study of the possible
new approximation proposed in this work, and for future studies in the theory.

