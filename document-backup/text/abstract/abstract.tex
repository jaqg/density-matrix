% ------
\section*{Abstract}
% ------
In this thesis, the application of Reduced Density Matrix 
Functional Theory (RDMFT) in the context of electronic structure calculations
is explored.
Therefore, the viability and 
accuracy of using one-particle (1-RDM) and two-particle reduced density 
matrices (2-RDM) in place of traditional wavefunction methods is investigated.
Both exact RDMs, computed from the integration of the wavefunction, and
approximated RDMs are studied.
The primary objective is to create a working environment that 
enables the optimization and verification of a potentially new parametrization
of the 2-RDM proposed by our group and explained within this document.

A significant advantage of RDMFT over conventional wavefunction-based methods 
is its ability to handle static correlation effects efficiently, making it a 
promising approach for systems where these effects are pronounced.
Additionally, the use of reduced density matrices simplifies the treatment of 
the kinetic energy term and avoids the need for introducing a fictitious non-interacting
system, as required in Density Functional Theory (DFT).

Despite its potential, the application of RDMFT is currently limited to small 
systems due to the computational complexity associated with larger systems.
This limitation stems from the intricate nature of constructing and managing 
higher-order density matrices and the challenges in ensuring their $N$-representability.

Future research directions include extending the methodologies proposed in 
this thesis to Time-Dependent Reduced Density Matrix Functional Theory (TDRDMFT),
which would enable the study of dynamic processes in electronic systems.
Moreover, significant theoretical advancements have been made over the past 
decades, particularly with the development of contracted Schrödinger equations 
and higher-order reduced density matrices, which promise to 
further enhance the applicability and robustness of RDMFT.

In conclusion, this work provides a comprehensive framework for employing 
RDMFT in electronic structure calculations, demonstrating its potential 
advantages and outlining the challenges that need to be addressed for broader application.

\vspace{0.3cm}
\textbf{Objectives:}
\begin{enumerate}[label={}]
    \item \textit{First:} Create a program integrated with the Dalton software for the obtenction
        of exact RDMs
    \item \textit{Second:} Validate available approximations for the electron repulsion functional
        in terms of the 2-RDM.
    \item \textit{Third:} Develop the formalism of a possible new approximation and propose possible
        parametrization and optimization methods.
    \item \textit{Foruth:} Create a framework for the construction, testing and deeper study of
        the 2-RDM, with the computation of RDM-derived properties, as the energy,
        comparable with the literature and/or software calculations, and
        direct metrics, as the Minkowski distance, for the direct comparison
        of RDMs.
\end{enumerate}
