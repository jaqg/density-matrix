\documentclass[12pt, a4paper]{article}

\usepackage[utf8]{inputenc}
\usepackage[T1]{fontenc}
\usepackage{textcomp}
\usepackage[spanish, es-tabla]{babel}
\usepackage{amsmath, amssymb}

\input{/home/jose/Documents/latex/preamble/paquetes1} %mis paquetes
% \usepackage[backend=biber,style=chem-acs,terseinits=true,sorting=none,isbn=false,doi=false]{biblatex}
% \addbibresource{/home/jose/Documents/latex/preamble/references} % extension must be written
%%%%
\usepackage[backend=bibtex,style=chem-acs,terseinits=true,sorting=none,isbn=false,doi=false]{biblatex}
% \usepackage[backend=bibtex,style=chem-acs,terseinits=true,sorting=none,isbn=false,doi=false]{biblatex}
\bibliography{/home/jose/Documents/latex/preamble/references}
%%%
\ExecuteBibliographyOptions{%
  citetracker=true,% Citation tracker enabled in order not to repeat citations, and have two lists.
  sorting=none,% Don't sort, just print in the order of citation
  alldates=long,% Long dates, so we can tweak them at will afterwards
  dateabbrev=false,% Remove abbreviations in dates, for same reason as ``alldates=long''
  articletitle=true,% To have article titles in full bibliography
  maxcitenames=999% Number of names before replacing with et al. Here, everyone.
  }

% No brackets around the number of each bibliography entry
\DeclareFieldFormat{labelnumberwidth}{#1\addperiod}

% Suppress article title, doi, url, etc. in citations
\AtEveryCitekey{%
  \ifentrytype{article}
    {\clearfield{title}}
    {}%
  \clearfield{doi}%
  \clearfield{url}%
  \clearlist{publisher}%
  \clearlist{location}%
  \clearfield{note}%
}

% Print year instead of date, when available; make use of urldate
\DeclareFieldFormat{urldate}{\bibstring{urlseen}\space#1}
\renewbibmacro*{date}{% Based on date bib macro from chem-acs.bbx
  \iffieldundef{year}
    {\ifentrytype{online}
       {\printtext[urldate]{\printurldate}}
       {\printtext[date]{\printdate}}}
    {\printfield[date]{year}}}

% Remove period from titles
\DeclareFieldFormat*{title}{#1}
% Make year bold for @book types
\DeclareFieldFormat[book]{date}{\textbf{#1}} % doctorate added this line
\DeclareFieldFormat[book]{title}{\textit{#1}} % doctorate added this line
\DeclareFieldFormat[book]{publisher}{#1,} % doctorate added this line
% Embed doi and url in titles, when available
\renewbibmacro*{title}{% Based on title bib macro from biblatex.def
  \ifboolexpr{ test {\iffieldundef{title}}
               and test {\iffieldundef{subtitle}} }
    {}
    {\ifboolexpr{ test {\ifhyperref}
                  and not test {\iffieldundef{doi}} }
       {\href{http://dx.doi.org/\thefield{doi}}
          {\printtext[title]{%
             \printfield[titlecase]{title}%
             \setunit{\subtitlepunct}%
             \printfield[titlecase]{subtitle}}}}
       {\ifboolexpr{ test {\ifhyperref}
                     and not test {\iffieldundef{url}} }
         {\href{\thefield{url}}
            {\printtext[title]{%
               \printfield[titlecase]{title}%
               \setunit{\subtitlepunct}%
               \printfield[titlecase]{subtitle}}}}
         {\printtext[title]{%
            \printfield[titlecase]{title}%
            \setunit{\subtitlepunct}%
            \printfield[titlecase]{subtitle}}}}%
     \newunit}%
  \printfield{titleaddon}%
  \clearfield{doi}%
  \clearfield{url}%
  \clearfield{pagetotal}%
  \clearlist{language}% doctorate added this
  \clearfield{note}% doctorate added this
  \ifentrytype{article}% Delimit article and journal titles with a period
    {\adddot}
    {}}
 %configuracion para l bibliografia
\input{/home/jose/Documents/latex/preamble/paquetesTikz} %mis paquetes
\input{/home/jose/Documents/latex/preamble/tikzstyle} %Estilo para las gráficas

\decimalpoint

% \graphicspath{{../figuras/}}

\author{José Antonio Quiñonero Gris}
\title{\textbf{Explorando el efecto túnel en la molécula de amoníaco}}

% \date{}

\date{\today}

\begin{document}
\maketitle

\begin{enumerate}[label=\textbf{Slide \arabic*.}]
    \setcounter{enumi}{1} % Empieza a contar por 2
    \item So, this is the table of contents.

        First, I will introduce you our work.
        Second, I will describe the inversion motion of the ammonia molecule
        Third, I will determine the stationary states
        Fourth, I will demostrate the creation of non-stationionary states and
        Fith, We will study its time dynamics
        Sixt, We will calculate the survival probability, expected values and \textit{visualize} tunneling
        And, finally, the conclusions.
    % ------
    \section{Introduction and objectives}
    % ------

    \item \textit{tunneling} is the name given to the penetration and transmission of molecules through potential energy barriers. On a microscopic scale, it is a purely quantum effect, of a probabilistic nature, which occurs in molecules whose energy is less than the barrier, which are transmitted through it through a classically forbidden region for them (red arrow). It is one of the most surprising and paradoxical phenomena of quantum mechanics that describes, in a coherent and formal way, a behavior that breaks with the corresponding classical description (blue arrow), giving quantum theory the explanatory power of the microscopic domain.

        The tunneling probability decreases exponentially with the width of the barrier, with the mass of the molecule, and with the square root of the height of the barrier relative to the energy of the molecule.

    \item It is present in all quantum systems and is used in scientific and technological fields such as:

         In chemistry, being of great importance in astrochemistry (in adsorption and desorption processes), electrochemistry, radioactive processes, catalysis and any chemical reaction that involves a transition state.

         In the same way, it is present in biological systems, such as proteins, and processes of interest such as proton transfer in DNA, which can cause mutations.

         And in technological fields such as quantum computing and devices such as the tunnel effect diode or the scanning tunneling microscope (STM).

    \item Our objectives in this work have been the following:
        \begin{enumerate}[label=\arabic*.]
            \item Study Quantum Tunneling in a realistic example, such as the inversion motion of \ch{NH3}.
            \item Exemplify the creation of non-stationary states as superpositions of stationary states
            \item Study time dynamics of these non-stationary states
            \item \textit{Visualize} tunneling, in real time, with the help of quantum mechanical simulations.
        \end{enumerate}

    % ------
    \section{Inversion motion of \ch{NH3}}
    % ------

    \item To study the inversion motion of the ammonia molecule, we define the inversion coordinate $x$ as the distance between the \ch{N} atom and the plane formed by the 3 \ch{H} atoms. That is, the height of the pyramid.

    \item The pyramidal ammonia molecule (with $C_{3v}$ symmetry) exhibits an umbrella-type vibrational motion in which the hydrogen atoms move vertically, leading to the inversion of the molecule, passing through a plane transition state (of symmetry $D_{3h}$).

        It is a high-amplitude vibrational motion that, although it does not correspond exactly to any normal mode of vibration, represents the angular symmetric or umbrella mode.

    \item According to this definition, the lower energy pyramidal equilibrium geometries (wells) are represented with two identical oscillators, so the potential is symmetric with respect to the reflection of the higher energy planar configuration (barrier). Therefore, the potential energy curve that describes this movement has the form of a double symmetric potential well, so that we consider the inversion motion of ammonia as an anharmonic vibrational motion of the \ch{NH3} molecule between the two potential wells.

    % ------
    \section{Determination of stationary states}
    % ------

    \item The first thing we need to study the tunnel effect are the energy levels and the eigenfunctions (stationary states) of the system. To do this, we have to solve the time-independent Schrödinger equation, where the Hamiltonian of our system is [equations], $\psi_i$ are the eigenfunctions and $E_i$ the eigenvalues or energy levels.

    \item To do this, we use the matrix formulation of the linear variational method, taking the set of eigenfunctions of the one-dimensional harmonic oscillator.

    \item By increasing the number of basis functions, the value of the variational energies decreases until the convergence given by the desired precission, which we consider as the energy levels of the system. In this figure, I present the variational energies for the first 4 energy levels versus the number of basis functions $N$. We can see that, as we expected, the value of the variational energies descreases by increasing $N$ until converged.

    \item In this table, I present the value of the energy levels, which can be seen in the figure, with the potential energy curve. We can appreciate the efect of the energy barrier on the quasi-degeneration of the first two vibrational levels ($v=0^{\pm}$), corresponding to the first doublet, and the levels corresponding to the second doublet ($v=1^{\pm}$).

        Also, we can see that these energy levels are below the height of the barrier. Namely, classicaly, an ammonia molecule in it's fundamental state or any of the first 3 excited states wouldn't be able to carry out the inversion motion, as it's energy is less than the energy needed (barrier height) for it to take place. It would need an external energy supply (for example, thermal energy) sufficient for its energy to be greater than the barrier, and thus be able to complete the inversion motion. Otherwise, the molecule would keep oscillating between the return points, around the equilibrium position of the well where it was.

    % \item A continuación, represento las funciones propias de los cuatro primeros niveles vibracionales frente a la coordenada de inversión $x$. Las funciones propias que caen por debajo de la barrera están deslocalizadas simétricamente respecto a la barrera en los dos pozos de potencial de acuerdo con las propiedades de simetría, siendo funciones pares las correspondientes a niveles pares,
    \item Next, I plot the eigenfunctions of the first four vibrational levels against the inversion coordinate $x$. The eigenfunctions that fall below the barrier are symmetrically delocalized with respect to the barrier in the two potential wells according to the symmetry properties, with even functions corresponding to even levels,

    % \item e impares las correspondientes a los niveles impares. Esta deslocalización simétrica de la densidad de probabilidad posicional en ambos pozos implica que existe la misma probabilidad de encontrar a la molécula de amoníaco, representada por un estado estacionario, en cualquiera de los dos pozos. Se dice entonces que la molécula está deslocalizada en los dos pozos. Es necesario crear estados no estacionarios localizados en uno de los pozos para seguir la evolución temporal mecanocuántica de los mismos y observar el efecto túnel.
    \item and odd those corresponding to odd levels.

        This symmetric delocalization of the positional probability density in both wells implies that there is an equal probability of finding the ammonia molecule, represented by a stationary state, in either of the two wells. The molecule is then said to be delocalized in the two wells. It is necessary to create localized non-stationary states in one of the wells to follow their quantum mechanical time evolution and to observe the tunnelling effect.

        % ------
        \section{Preparation of non-stationary states}
        % ------

    % \item Entendemos por estado no estacionario un estado de probabilidad espacial cuántica en \textit{movimiento} formado por una \textit{superposición} de funciones de onda de estados estacionarios de diferentes energías.
    \item We understand by non-stationary state a state of quantum space probability in \textit{motion} formed by a \textit{superposition} of wavefunctions of stationary states of different energies.

    % Una vez hemos determinado los estados estacionarios de nuestro sistema, podemos formar combinaciones lineales (superposiciones) de estos para crear estados no estacionarios, cuya energía es la media de las energías de los estados que se encuentran en superposición. En nuestro caso, formamos una superposición de los estados estacionarios $\Ket{\psi_0}$, $\Ket{\psi_1}$, $\Ket{\psi_2}$, $\Ket{\psi_3}$ correspondientes a los cuatro primeros niveles de energía. Le damos valores a $\alpha$ entre $0$ y $90^{\circ}$ para formar estados no estacionarios por debajo de la barrera.
    Once we have determined the stationary states of our system, we can form linear combinations (superpositions) of these to create non-stationary states, whose energy is the mean of the energies of the states that are in superposition. In our case, we form a superposition of the stationary states $\Ket{\psi_0}$, $\Ket{\psi_1}$, $\Ket{\psi_2}$, $\Ket{\psi_3}$ corresponding to the first four energy levels. We give vales to $\alpha$ between $0$ and $90^{\circ}$ to form non-stationary states below the barrier.

    % \item De esta manera, los estados no estacionarios iniciales límite, para $\alpha=0$ y $\alpha=90^{\circ}$, están formados por la superposición de los estados estacionarios correspondientes a los dos dobletes, respectivamente. Vemos cómo estos estados se encuentran localizados en el pozo de la izquierda.
    \item In this way, the limit initial non-stationary states, for $\alpha=0$ and $\alpha=90^{\circ}$, are formed by the superposition of the stationary states corresponding to the two doublets, respectively. We see how these states are located in the well on the left.

    %     De la misma manera, hemos creado estados no estacionarios intermedios para distintos valores de $\alpha$, aunque no los incluyo en esta presentación.
        In the same way, we have created intermediate non-stationary states for different values of $\alpha$, although I do not include them in this presentation.

        % ------
        \section{Time dynamics of non-stationary states}
        % ------

    % \item Para ver cómo evolucionan con el tiempo estos estados no estacionarios tenemos que resolver la \textit{ecuación de Schrödinger dependiente del tiempo}, donde la función de onda completa que satisface dicha ecuación está formada por una superposición de estados estacionarios, incluyendo ahora la dependencia temporal.
    \item To see how these non-stationary states evolve over time, we need to solve the \textit{time-dependent Schrödinger equation}, where the complete wave function that satisfies this equation is formed by a superposition of stationary states, now including the time dependence.

    % \item Aunque ahora incluyamos la componente temporal en los estados estacionarios, el valor esperado de un observable para estos estados no depende del tiempo. Es por esta razón que creamos una superposición de estados no estacionarios de diferente energía.
    \item Although we now include the time dependence in stationary states, the expectation value of an observable for these states does not depend on time. It is for this reason that we create a superposition of non-stationary states of different energy.

    %     Calculamos el valor esperado de la posición como [ecuación].
        We calculate the expectation value of the position as [equation].

    %     La probabilidad de supervivencia $P_s$ se define como la probabilidad de que el estado no estacionario coincida con el estado inicial y viene dada, por tanto, por [ecuación].
        The survival probability $P_s$ is defined as the probability that the non-stationary state coincides with the initial state and is therefore given by [equation].

    %     Se define el tiempo de recurrencia $t_r$ como el tiempo que tarda el estado no estacionario en volver a su estado inicial. A partir de la definición de probabilidad de supervivencia, encontramos que el tiempo de recurrencia viene dado por [ecuación], que para los estados no estacionarios límite vale [tabla]. Vemos cómo al aumentar la anchura y altura de la barrera en el caso límite $\alpha=0$, el tiempo de recurrencia es mucho mayor que en caso límite correspondiente a $\alpha=90^{\circ}$
        The recurrence time $t_r$ is defined as the time it takes for the non-stationary state to return to its initial state. From the definition of survival probability, we find that the recurrence time is given by [equation], which for limit non-stationary states is equal to [table]. We see how by increasing the width and height of the barrier in the limit case $\alpha=0$, the recurrence time is much greater than that in the limit case corresponding to $\alpha=90^{\circ}$

    % \item A continuación, represento la probabilidad de supervivencia para los estados no estacionarios iniciales correspondientes a $\alpha=0,15,30^{\circ}$.
    \item Next, I plot the survival probability for the initial non-stationary states corresponding to $\alpha=0,15,30^{\circ}$.

    %     Como esperábamos de la definición de $P_s$, la probabilidad de supervivencia sigue una periodicidad. El tiempo para el que la probabilidad de supervivencia se hace máxima se corresponde con el tiempo de recurrencia. Lo primero que identificamos es que la probabilidad de supervivencia varía de una manera regular para el estado no estacionario límite y que, al aumentar $\alpha$, aparecen oscilaciones en los valores de $P_s(t)$ que se van haciendo mayores conforme aumenta $\alpha$.
         As expected from the definition of $P_s$, the survival probability follows a periodicity. The time for which the probability of survival becomes maximum corresponds to the time of recurrence. The first thing we identify is that the survival probability varies in a regular way for the limit non-stationary state and that, as $\alpha$ increases, oscillations appear in the values of $P_s(t)$ that become larger as it increases $\alpha$.

    %     Para observar mejor dichas oscilaciones, reduzco la escala de tiempo a 1 picosegundo.
         To better observe these oscillations, I reduce the time scale to 1 picosecond.

    % \item Teniendo en mente la definición de probabilidad de supervivencia, estas oscilaciones o cambios drásticos de $P_s(t)$ en pequeños intervalos implican que la probabilidad de encontrar al estado no estacionario en su situación inicial varía drásticamente en un intervalo de tiempo muy pequeño.
    \item Bearing in mind the definition of survival probability, these oscillations or drastic changes of $P_s(t)$ in small intervals imply that the probability of finding the non-stationary state in its initial situation varies drastically in a very small time interval.

    %     Si nos fijamos, por ejemplo, en la gráfica para $\alpha = 45^{\circ}$ en la región en torno a $t = 0.5$ ps, vemos cómo la probabilidad de supervivencia varía entre $\sim 0$ y $\sim 1$ en un pequeño intervalo de tiempo. Puede ser que alguno de estudes se esté preguntando: ¿se corresponde este resultado, por tanto, a que el estado no estacionario se encuentra oscilando de un pozo a otro en intervalos de tiempo muy pequeños? Más bien, la interpretación de estos resultados implica que la totalidad del estado no estacionario no \textit{penetra} la barrera siempre que colisiona con la misma en estos estados no estacionarios intermedios, a diferencia de lo que ocurre en los estados no estacionarios límite, encontrándose deslocalizado entre los dos pozos. En el caso de los estados no estacionarios intermedios (por ejemplo, para $\alpha = 45^{\circ}$), el estado se desplaza desde el pozo de la izquierda, colisiona con la barrera, y sólo una pequeña parte (baja densidad de probabilidad) consigue penetrar la barrera y transmitirse hasta el pozo de la derecha, mientras el resto del estado continúa oscilando en el pozo de la izquierda. Conforme pasa el tiempo, el estado no estacionario habrá colisionado con la barrera un número de veces suficiente como para transmitirse al otro pozo (la densidad de probabilidad en el pozo de la derecha será mayor que en el de la izquierda) y podrá repetirse el proceso para su vuelta a la situación inicial. Esta deslocalización del estado explica las oscilaciones en los valores de $P_s(t)$.
        If we look, for example, at the graph for $\alpha = 45^{\circ}$ in the region around $t = 0.5$ ps, we see how the survival probability varies between $\sim 0$ and $ \sim 1$ in a small time interval. It may be that some of you are wondering: does this result correspond, therefore, to the fact that the non-stationary state is oscillating from one well to another in very small time intervals? Rather, the interpretation of these results implies that the entire non-stationary state does not \textit{penetrate} the barrier whenever it collides with it in these intermediate non-stationary states, unlike in the limit non-stationary states, which are delocalized between the two wells. In the case of intermediate non-stationary states (for example, for $\alpha = 45^{\circ}$), the state moves from the left well, collides with the barrier, and only a small part (low probability density) manages to penetrate the barrier and be transmitted to the well on the right, while the rest of the state continues to oscillate in the well on the left. As time goes by, the non-stationary state will have collided with the barrier enough times to be transmitted to the other well (the probability density in the well on the right will be higher than that in the one on the left) and the process can be repeated for its return to the initial situation. This delocalization of the state explains the oscillations in the values of $P_s(t)$.

    %     Para poder entender mejor estas observaciones, calculamos el valor esperado de la posición y su evolución temporal
        In order to better understand these observations, we calculate the expectation value of the position and its time evolution

    % \item Las líneas horizontales negras [señalar] se corresponden a los puntos de corte y las líneas grises a la posición de los pozos. Nos interesan especialmente las líneas negras más gruesas, ya que se corresponden a los puntos de corte con la barrera.
    \item The black horizontal lines [point to] correspond to the cutoff points and the gray lines to the position of the wells. We are especially interested in the thicker black lines, since they correspond to the intersection points with the barrier.

    %     Al igual que con la probabilidad de supervivencia, conforme aumenta $\alpha$ aparecen oscilaciones en el valor esperado de la posición del estado. En el caso límite corespondiente a $\alpha = 0$, vemos cómo la evolución temporal del valor esperado de $x$ es regular, lo que nos permite imaginar que el movimiento entre los pozos de este estado no estacionario es regular. Es decir, que la totalidad del estado penetra y se transmite a través de la barrera al chocar con la misma.  Sin embargo, como ya adelantábamos con la probabilidad de supervivencia, esto no es así para los estados no estacionarios intermedios. Las variaciones en el valor esperado de $x$ nos indican que el estado no estacionario se encuentra deslocalizado.
        Just like with the survival probability, as $\alpha$ increases, oscillations appear in the expectation value of the position of the state. In the limit case corresponding to $\alpha = 0$, we see how the time evolution of the expectation value of $x$ is regular, which allows us to imagine that the motion between the wells of this non-stationary state is regular. That is, the entire state penetrates and is transmitted through the barrier upon colliding with it. However, as we already anticipated with the survival probability, this is not the case for intermediate non-stationary states. Variations in the expectation value of $x$ indicate that the non-stationary state is delocalized.

    %     De nuevo, reduzco la escala temporal a 1 ps.
        Again, I reduce the timescale to 1ps.

    % \item Vemos cómo, para el estado no estacionario límite correspondiente a  $\alpha=90^{\circ}$, obtenemos el resultado esperado. Sin embargo, para $\alpha = 45^{\circ}$, el valor esperado de la posición para esta escala temporal presenta oscilaciones y es menor de cero (posición del máximo de la barrera), variando en torno al punto de corte con la barrera (línea horizontal punteada en torno a $\approx -0.2$ \AA). Es decir, el estado no estacionario se encuentra oscilando en el pozo de la izquierda, chocando con la barrera y entrando en ella levemente. Para un tiempo suficiente, como podemos apreciar en la gráfica para $\alpha = 45^{\circ}$ de la figura anterior, el estado termina transmitiéndose al otro pozo, y completando el movimiento de inversión. Para el resto de estados no estacionarios intermedios sucede algo análogo, sólo que con oscilaciones menores en el valor esperado de la posición.
    \item We see how, for the limit non-stationary state corresponding to $\alpha=90^{\circ}$, we obtain the expected result. However, for $\alpha = 45^{\circ}$, the expectation value of the position for this time scale shows oscillations and is less than zero (position of the maximum of the barrier), varying around the cutoff point with the barrier (dotted horizontal line around $\approx -0.2$ \AA). That is, the non-stationary state is found oscillating in the well on the left, colliding with the barrier and penetrating it slightly. For a sufficient time, as we can see in the graph for $\alpha = 45^{\circ}$ of the previous figure, the state ends up being transmitted to the other well, and completing the inversion motion. For the rest of the intermediate non-stationary states something similar happens, only with smaller oscillations in the expected value of the position.

    % \item Consideramos que la mejor forma de entender este efecto es visualizándolo, por lo que hemos creado una serie de simulaciones mecanocúanticas de la evolución temporal de dichos paquetes no estacionarios. A continuación, presento las correspondientes a los estados límite. En ellas, vemos cómo la densidad de probabilidad en el pozo de la derecha aumenta con el tiempo, a causa de la transmisión de la molécula de amoníaco por efecto túnel.
    \item We believe that the best way to understand this effect is to visualize it, so we have created a series of quantum mechanical simulations of the time evolution of these non-stationary states. Next, I present those corresponding to the limit states. In them, we see how the probability density in the well on the right increases with time, due to the tunneling of the ammonia molecule.

    \item \textbf{Conclusions:}
        \textbf{First}. Tunneling is a purely quantum effect, difficult to understand, whose visualization facilitates its understanding.

        \textbf{Second}. Quantum Mechanics tells us that a system has a series of eigenvalues of energy and eigenfunctions, which we obtain by solving the Schrödinger equation for a characteristic Hamiltonian operator. However, the system can also be described, as shown in this work, by non-stationary states formed as superpositions of stationary states of different energies. Non-stationary states can be created experimentally using spectroscopic techniques, and only when a measurement is made, the states superposition collapses into the eigenstate obtained from the measured magnitude.

        \textbf{Third}. The expectation values of an observable corresponding to the eigenfunctions of the Hamiltonian, the stationary states, do not vary with time as it classically occurs. To do this, it is necessary to create a non-stationary state formed by the superposition of stationary states of different energies.

         \textbf{Fourth.} In the inversion motion of the ammonia molecule, described by two identical potential wells corresponding to the equilibrium pyramidal geometries of the molecule, separated by a potential barrier corresponding to the planar geometry of the molecule, the stationary states that fall below the barrier are symmetrically delocalized with respect to the barrier and there is no possibility that they evolve in time.

        \textbf{Fifth.} To observe tunneling in the inversion motion of the ammonia molecule, it is therefore necessary to even create localized non-stationary states in one of the wells and follow their quantum mechanical time evolution.

        \textbf{Sixth.} The initial non-stationary states are constructed as a linear combination of stationary states that allows us to localize that state in one of the potential wells.

        \textbf{Seventh.} The initial non-stationary states localized in one of the potential wells evolve over time crossing the barrier and moving to the other potential well by tunneling. This movement is uniform for the initial non-stationary states formed by the superposition of two stationary states and irregular for the combination of more than two stationary states, causing the state to oscillate around its initial equilibrium position. That is, the tunneling probability is greater for non-stationary states formed by the superposition of two stationary states than for non-stationary states constructed as a linear combination of more than two stationary states.
\end{enumerate}

\end{document}
